\chapter{\ifcpe บทสรุปและข้อเสนอแนะ\else Conclusions and Discussion\fi}

\section{\ifcpe สรุปผล\else Conclusions\fi}

ทางผู้พัฒนาได้เล็งเห็นถึงความสำคัญของวิชาวิทยาการคำนวณเลยได้ศึกษาวิชาวิทยาการคำนวณซึ่งเป็นวิชาล่าสุดที่ถูกนำมาเพิ่มในรายวิชาวิทยาศาสตร์ในปี พศ. 2560 โดยหลักๆ ในตัวรายวิชาจะเป็นเกี่ยวกับ
เทคโนโลยี ประกอบไปด้วย การเขียนโปรแกรม, การใช้เทคโนโลยี, การมีภูมิคุ้มกันต่อเทคโนโลยี, การพัฒนาทักษะการคิดอย่างเป็นขั้นเป็นตอน, ฯลฯ
และเมื่อลงพื้นที่เพื่อไปศึกษาวิชาวิทยาการคำนวณ ณ โรงเรียนบ้านดอยเต่าซึ่งเป็นโรงเรียนรอบนอก ทำให้เราได้เล็งเห็นปัญหาของตัวเด็กนักเรียนเกี่ยวกับวิชาวิทยาการคำนวณ นั่นก็คือ
การที่ตัวเด็กนักเรียนเองไม่สามามรถเรียนรู้วิชาวิทยาการคำนวณได้เต็มที่ในหลายๆ ปัจจัยที่เกี่ยวข้อง เช่นการที่เด็กไม่ชอบเรียนวิชาวิทยาการคำนวณ
เพราะว่าไม่มีครูสอนหรือรวมไปถึงตัวเด็กเองไม่ได้ตั้งใจเรียนอยู่แล้ว ทำให้เกิดโครงงานนี้มาเพื่อเป็นสื่อและเครื่องมือผ่านเกมที่เป็น Mobile เกมผ่านโทรศัพท์มือถือเพื่อการพัฒทักษะวิชาวิทยาการคำนวณ
โดยที่ผู้พัฒนาจะมุ่งเน้นไปที่ทักษะการคิดอย่างเป็นขั้นเป็นตอนซึ่งเป็นทักษะที่สำคัญในการแก้ไขปัญหาต่างๆ
โครงงานนี้มีจุดประสงค์ดังต่อไปนี้คือ ลดความเหลื่อมล้ำทางด้านการศึกษา, ทำให้เด้กนักเรียนสามารถนำความรู้ที่ได้ไปแก้ไขปัญหาในชีวิตประจำวันได้ และ
ผลักดันหลุกสูตรวิชาวิทยาการคำนวณ เนื้อหาในตัวเกมที่ผู้พัฒนาสร้างขึ้นจะเน้นไปที่การพัฒนาทักษะการคิดอย่างเป็นขั้นเป็นตอน
ซึ่งผู้พัฒนาจะอิงตามหลักสูตรวิชาวิทยาการคำนวณของ สสวท. โดยเนื้อหาในตัวเกมนั้นจะเป็นในรูปแบบของการแก้ไขปริศนากล่าวคือ ผู้เล่น(เด็กนักเรียน)
ต้องแก้ปริศนาคือการให้ตัวละครในเกมเดินไปยังจุดหมายหรือเส้นชัยผ่านการเลือกวาง Blockly Code ซึ่งรูปแบบของด่านในเกม
จะอิงตามหนังสือวิชาวิทยาการคำนวณของ สสวท. โดยผู้พัฒนาได้แบ่งการทดสอบเป็น 2 อย่าง คือการทดสอบกับกลุ่มตัวอย่างซึ่งก็คือนักเรียนชั้นประถมศึกษาปีที่ 3-6 จำนวณ 71 คนเพื่อทดสอบ
ว่าเครื่องมือหรือเกมเสริมทักษะวิชาวิทยาการคำนวณนั้นสามารถเสริมทักษะวิชาวิทยาการคำนวณได้จริงผ่านการทดสอบ pre-test, post-test โดยข้อสอบจะอิงตามหนังสือวิชาวิทยาการคำนวณของ สสวท. ซึ่งผลที่ได้ออกมาคือนักเรียนทุกคนนั้นมีพัฒนาการหลังจากการ
เล่นเกมเสริมทักษะวิชาวิทยาการคำนวน เนื่องจากโครงงานนี้เป็นการสร้างซอฟแวร์ซึ่งเป็นในรูปแบบของเกมทำให้ต้องมีการทดสอบระบบต่างๆ ซึ่งประกอบไปด้วย UX, UI, เนื้อหาในเกม, ลำดับความยากง่ายของด่านต่อไป เป็นต้น ดังนั้น
ทางผู้พัฒนาเลยมีการทดสอบตัวระบบของเกมเสริมทักษะวิชาวิทยาการคำนวณผ่านผู้เชี่ยวชาญจำนวณ 3 ท่านซึ่งมีประสบการทางด้านการเขียนโปรแกรมผ่าน Unity ไม่ต่ำกว่า 3 ปี โดยผู้พัฒนาใช้โมเดล IOC ในการวัดผลและผลที่ออกมา
คือทุกหัวข้อในการทดสอบผ่านทั้งหมด ดังนั้นจึงสรุปได้ว่าโครงงานเกมเสริมทักษะวิชาวิทยาการคำนวณนั้นสามารถลดความเหลื่อมล้ำทางด้านการศึกษากล่างคือ หลังจากนักเรียนได้เล่นเกมเสริมทักษะวิชาวิทยาการคำนวณไป
นักเรียนสามารถทำข้อสอบวิชาวิทยาการคำนวณที่อิงจากหนังสือ สสวท. ได้ซึ่งไม่ต่างกับเด็กนักเรียนที่เรียนตามหนังสือวิชาวิทยาการคำนวณของ สสวท. 
นอกจากนี้นักเรียนเองสามารถนำความรู้ไปแก้ไขปัญหาในชีวิตจริงได้กล่าวคือ ใน post-test ทางผู้พัฒนาได้เพิ่มความลึกของข้อสอบโดย
เพิ่มการคิดต่อยอดเข้าไปซึ่งตัวเด็กนักเรียนเองก็สามารถทำได้ หลังจากที่ผู้พัฒนาได้เข้าไปลงพื้นที่เพื่อทดสอบเกมเสริมทักษะวิชาวิทยาการคำนวณกับกลุ่มตัวอย่างทำให้ทางโรงเรียนที่ไปลงพื้นที่นั้น
สนใจในเกมเสริมทักษะวิชาวิทยาการคำนวณเป็นอย่างมากอละได้เล็งเห็นถึงความสำคัญของหลักสูตรวิชาวิทยาการคำนวณซึ่งเป็นการผลักดันหลักสูตรวิชาวิทยาการคำนวณตามที่ผู้พัฒนาได้ตั้งไว้

\section{\ifcpe ปัญหาที่พบและแนวทางการแก้ไข\else Challenges\fi}

ในการทำโครงงานนี้ พบว่าเกิดปัญหาหลักๆ อยู่ 3 ส่วน ดังนี้

\subsection{ปัญหาระหว่างการทดสอบกับกลุ่มตัวอย่าง}
เมื่อทดสอบกับกลุ่มตัวอย่างสิ่งที่เรียกว่าบัคนั้นได้เกิดขึ้นจากการเขียนโปรแกรมของทางผู้พัฒนา เช่น เดิมตกแมพ, ตัวละครบินได้, ตัวแสดงผลหน้าเว็ปนั้นค้าง
รวมไปถึงการเข้าไปทดสอบในช่วงที่มีโรงระบาด (Covid-19) ทำให้มีอุปสรรคเช่น การเข้าใกล้ชิดกับนักเรียนและการรวมกลุ่มกันของนักเรียน
\subsection{ปัญหาทางด้านเครื่องมือสร้างเกม(Unity)}
เกมเสริมทักษะวิชาวิทยาการคำนวนที่ผู้พัฒนาได้พัฒนาขึ้นมานั้นเป็นเกมที่ไม่เหมือนเกมทั่วไปที่สร้างโดย Unity โดยตัวเกมหลักๆ
จะประกอบไปด้วยกันอยู่ 2 ส่วนคือส่วนที่เป็น Unity แสดงผลต่างๆ และหน้าเว็ป ซึ่งทางหน้าเว็ปผู้พัฒนาต้องเก็บไว้ในตัว Resources ไฟล์
บน Unity และเมื่อเริ่มการ Build โปรเจคที่โครงสร้างไฟล์ต่างจากปกติทำให้ Gradle Linker ทำงานผิดพลาด ดังนั้นผู้พัฒนาเลยต้องทำการ Manual Link ไฟล์ต่างๆ เอง
เลยเป็นส่วนที่ทำให้เสียเวลาในการพัฒนาตัวเกมมาก

\section{\ifcpe%
ข้อเสนอแนะและแนวทางการพัฒนาต่อ
\else%
Suggestions and further improvements
\fi
}
ข้อเสนอแนะเพื่อพัฒนาโครงงานนี้ต่อไป มีดังนี้
