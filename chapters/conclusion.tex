\chapter{\ifcpe บทสรุปและข้อเสนอแนะ\else Conclusions and Discussion\fi}

\section{\ifcpe สรุปผล\else Conclusions\fi}
\GBreply{เอาไปไว้ใน Intro มั้ย?}
% ทางผู้พัฒนาได้เล็งเห็นถึงความสำคัญของวิชาวิทยาการคำนวณเลยได้ศึกษาวิชาวิทยาการคำนวณซึ่งเป็นวิชาล่าสุดที่ถูกนำมาเพิ่มในรายวิชาวิทยาศาสตร์ในปี พศ. 2560 โดยหลักๆ ในตัวรายวิชาจะเป็นเกี่ยวกับ
% เทคโนโลยี ประกอบไปด้วย การเขียนโปรแกรม, การใช้เทคโนโลยี, การมีภูมิคุ้มกันต่อเทคโนโลยี, การพัฒนาทักษะการคิดอย่างเป็นขั้นเป็นตอน, ฯลฯ
% และเมื่อลงพื้นที่เพื่อไปศึกษาวิชาวิทยาการคำนวณ ณ โรงเรียนบ้านดอยเต่าซึ่งเป็นโรงเรียนรอบนอก ทำให้เราได้เล็งเห็นปัญหาของตัวเด็กนักเรียนเกี่ยวกับวิชาวิทยาการคำนวณ นั่นก็คือ
% การที่ตัวเด็กนักเรียนเองไม่สามามรถเรียนรู้วิชาวิทยาการคำนวณได้เต็มที่ในหลายๆ ปัจจัยที่เกี่ยวข้อง เช่นการที่เด็กไม่ชอบเรียนวิชาวิทยาการคำนวณ
% เพราะว่าไม่มีครูสอนหรือรวมไปถึงตัวเด็กเองไม่ได้ตั้งใจเรียนอยู่แล้ว ทำให้เกิดโครงงานนี้มาเพื่อเป็นสื่อและเครื่องมือผ่านเกมที่เป็น Mobile เกมผ่านโทรศัพท์มือถือเพื่อการพัฒทักษะวิชาวิทยาการคำนวณ
% โดยที่ผู้พัฒนาจะมุ่งเน้นไปที่ทักษะการคิดอย่างเป็นขั้นเป็นตอนซึ่งเป็นทักษะที่สำคัญในการแก้ไขปัญหาต่างๆ
% โครงงานนี้มีจุดประสงค์ดังต่อไปนี้คือ ลดความเหลื่อมล้ำทางด้านการศึกษา, ทำให้เด้กนักเรียนสามารถนำความรู้ที่ได้ไปแก้ไขปัญหาในชีวิตประจำวันได้ และ
% ผลักดันหลุกสูตรวิชาวิทยาการคำนวณ เนื้อหาในตัวเกมที่ผู้พัฒนาสร้างขึ้นจะเน้นไปที่การพัฒนาทักษะการคิดอย่างเป็นขั้นเป็นตอน
% ซึ่งผู้พัฒนาจะอิงตามหลักสูตรวิชาวิทยาการคำนวณของ สสวท. โดยเนื้อหาในตัวเกมนั้นจะเป็นในรูปแบบของการแก้ไขปริศนากล่าวคือ ผู้เล่น(เด็กนักเรียน)
% ต้องแก้ปริศนาคือการให้ตัวละครในเกมเดินไปยังจุดหมายหรือเส้นชัยผ่านการเลือกวาง Blockly Code ซึ่งรูปแบบของด่านในเกม
% จะอิงตามหนังสือวิชาวิทยาการคำนวณของ สสวท. โดยผู้พัฒนาได้แบ่งการทดสอบเป็น 2 อย่าง คือการทดสอบกับกลุ่มตัวอย่างซึ่งก็คือนักเรียนชั้นประถมศึกษาปีที่ 3-6 จำนวณ 71 คนเพื่อทดสอบ
% ว่าเครื่องมือหรือเกมเสริมทักษะวิชาวิทยาการคำนวณนั้นสามารถเสริมทักษะวิชาวิทยาการคำนวณได้จริงผ่านการทดสอบ pre-test, post-test โดยข้อสอบจะอิงตามหนังสือวิชาวิทยาการคำนวณของ สสวท. ซึ่งผลที่ได้ออกมาคือนักเรียนทุกคนนั้นมีพัฒนาการหลังจากการ
% เล่นเกมเสริมทักษะวิชาวิทยาการคำนวน เนื่องจากโครงงานนี้เป็นการสร้างซอฟแวร์ซึ่งเป็นในรูปแบบของเกมทำให้ต้องมีการทดสอบระบบต่างๆ ซึ่งประกอบไปด้วย UX, UI, เนื้อหาในเกม, ลำดับความยากง่ายของด่านต่อไป เป็นต้น ดังนั้น
% ทางผู้พัฒนาเลยมีการทดสอบตัวระบบของเกมเสริมทักษะวิชาวิทยาการคำนวณผ่านผู้เชี่ยวชาญจำนวณ 3 ท่านซึ่งมีประสบการทางด้านการเขียนโปรแกรมผ่าน Unity ไม่ต่ำกว่า 3 ปี โดยผู้พัฒนาใช้โมเดล IOC ในการวัดผลและผลที่ออกมา
% คือทุกหัวข้อในการทดสอบผ่านทั้งหมด ดังนั้นจึงสรุปได้ว่าโครงงานเกมเสริมทักษะวิชาวิทยาการคำนวณนั้นสามารถลดความเหลื่อมล้ำทางด้านการศึกษากล่างคือ หลังจากนักเรียนได้เล่นเกมเสริมทักษะวิชาวิทยาการคำนวณไป
% นักเรียนสามารถทำข้อสอบวิชาวิทยาการคำนวณที่อิงจากหนังสือ สสวท. ได้ซึ่งไม่ต่างกับเด็กนักเรียนที่เรียนตามหนังสือวิชาวิทยาการคำนวณของ สสวท. 
% นอกจากนี้นักเรียนเองสามารถนำความรู้ไปแก้ไขปัญหาในชีวิตจริงได้กล่าวคือ ใน post-test ทางผู้พัฒนาได้เพิ่มความลึกของข้อสอบโดย
% เพิ่มการคิดต่อยอดเข้าไปซึ่งตัวเด็กนักเรียนเองก็สามารถทำได้ หลังจากที่ผู้พัฒนาได้เข้าไปลงพื้นที่เพื่อทดสอบเกมเสริมทักษะวิชาวิทยาการคำนวณกับกลุ่มตัวอย่างทำให้ทางโรงเรียนที่ไปลงพื้นที่นั้น
% สนใจในเกมเสริมทักษะวิชาวิทยาการคำนวณเป็นอย่างมากอละได้เล็งเห็นถึงความสำคัญของหลักสูตรวิชาวิทยาการคำนวณซึ่งเป็นการผลักดันหลักสูตรวิชาวิทยาการคำนวณตามที่ผู้พัฒนาได้ตั้งไว้
จากผลการทดสอบจากบทที่ 4 เกมเสริมทักษะวิชาวิทยาการคำนวณเป็นเกมที่สามารถเพิ่มทักษะการคิดเชิงคำนวณและ การคิดอย่างเป็นขั้นเป็นตอนได้ดีเยี่ยมซึ่งที่จะพึงมีและควรค่าที่จะพัฒนาต่อไป
ซึ่งบรรลุจุดประสงค์ที่ได้ตั้งไว้นั่นคือ การลดความเหลือมล้ำทางด้านการศึกษากล่าวคือหลังจากนักเรียนได้เล่นเกมเสริมทักษะวิชาวิทยาการคำนวณไปแล้วนักเรียนสามารถทำข้อสอบวิชาวิทยาการคำนวณที่อิงจากหนังสือ สสวท. ได้ไม่ต่างกับเด็กนักเรียนที่เรียนตามหนังสือวิชาวิทยาการคำนวณของ สสวท. 
นอกจากนี้นักเรียนเองสามารถนำความรู้ไปแก้ไขปัญหาในชีวิตจริงได้กล่าวคือ ใน post-test ทางผู้พัฒนาได้เพิ่มความลึกของข้อสอบโดย
เพิ่มการคิดต่อยอดเข้าไปซึ่งตัวเด็กนักเรียนเองก็สามารถทำได้ หลังจากที่ผู้พัฒนาได้เข้าไปลงพื้นที่เพื่อทดสอบเกมเสริมทักษะวิชาวิทยาการคำนวณกับกลุ่มตัวอย่างทำให้ทางโรงเรียนที่ไปลงพื้นที่นั้น
สนใจในเกมเสริมทักษะวิชาวิทยาการคำนวณเป็นอย่างมากและได้เล็งเห็นถึงความสำคัญของหลักสูตรวิชาวิทยาการคำนวณซึ่งเป็นการผลักดันหลักสูตรวิชาวิทยาการคำนวณตามที่ผู้พัฒนาได้ตั้งไว้


\section{\ifcpe ปัญหาที่พบและแนวทางการแก้ไข\else Challenges\fi}

ในการทำโครงงานนี้ พบว่าเกิดปัญหาหลักๆ อยู่ 2 ส่วน ดังนี้

\subsection{ปัญหาทางด้านเครื่องมือสร้างเกม(Unity)}
เกมเสริมทักษะวิชาวิทยาการคำนวนที่ผู้พัฒนาได้พัฒนาขึ้นมานั้นเป็นเกมที่ไม่เหมือนเกมทั่วไปที่สร้างด้วย \newline
Unity โดยตัวเกมหลักๆจะประกอบไปด้วยกันอยู่ 2 ส่วนคือส่วนที่เป็น Unity แสดงผลต่างๆ และหน้าเว็ป ซึ่งทางหน้าเว็ปผู้พัฒนาต้องเก็บไว้ในตัว Resources ไฟล์
บน Unity และเมื่อเริ่มการ Build โปรเจคที่โครงสร้างไฟล์ต่างจากปกติทำให้ Gradle Linker ทำงานผิดพลาด ดังนั้นผู้พัฒนาเลยต้องทำการ Manual Link ไฟล์ต่างๆ เอง
เลยเป็นส่วนที่ทำให้เสียเวลาในการพัฒนาตัวเกมมาก
\subsection{ปัญหาระหว่างการทดสอบกับกลุ่มตัวอย่าง}
ปัญหาระหว่างการทดสอบแบ่งได้ออกเป็น 2 หัวข้อ ดังนี้
% \subsubsection{เมื่อทดสอบกับกลุ่มตัวอย่างสิ่งที่เรียกว่าบัคนั้นได้เกิดขึ้นจากการเขียนโปรแกรมของทางผู้พัฒนา เช่น เดิมตกแมพ, ตัวละครบินได้, ตัวแสดงผลหน้าเว็ปนั้นค้าง ซึ่งได้รับการแก้ไข้หลังจากไปทดสอบกับกลุ่มตัวอย่างเสร็จแล้ว}
% \subsubsection{การเข้าไปทดสอบในช่วงที่มีโรคระบาด Covid-19 ทำให้มีอุปสรรคเช่น การเข้าใกล้ชิดกับนักเรียนและการรวมกลุ่มกันของนักเรียน ทำให้ต้องมีขั้นตอนในการปฎิบัติตามมาตรการป้องกันคือ การตรวจหาเชื้อ Covid-19 ก่อนเข้าไปทดสอบกับกลุ่มตัวอย่างและต้องทดสอบกับกลุุ่มตัวอย่างทีละ 2-3 คน}
\begin{enumerate}
    \item เมื่อทดสอบกับกลุ่มตัวอย่างสิ่งที่เรียกว่าบัคนั้นได้เกิดขึ้นจากการเขียนโปรแกรมของทางผู้พัฒนา เช่น เดิมตกแมพ, ตัวละครบินได้, ตัวแสดงผลหน้าเว็ปนั้นค้าง ซึ่งได้รับการแก้ไข้หลังจากไปทดสอบกับกลุ่มตัวอย่างเสร็จแล้ว
    \item การเข้าไปทดสอบในช่วงที่มีโรคระบาด Covid-19 ทำให้มีอุปสรรคเช่น การเข้าใกล้ชิดกับนักเรียนและการรวมกลุ่มกันของนักเรียน ทำให้ต้องมีขั้นตอนในการปฎิบัติตามมาตรการป้องกันคือ การตรวจหาเชื้อ Covid-19 ก่อนเข้าไปทดสอบกับกลุ่มตัวอย่างและต้องทดสอบกับกลุุ่มตัวอย่างทีละ 2-3 คน
\end{enumerate}

\section{\ifcpe%
ข้อเสนอแนะและแนวทางการพัฒนาต่อ
\else%
Suggestions and further improvements
\fi
}
ข้อเสนอแนะเพื่อพัฒนาโครงงานนี้ต่อไป มีดังนี้
\subsection{การนำเอาไปใช้เพื่อเป็นเครื่องมือในการต่อยอดและพัฒนาทักษะทางด้านการคิดเชิงคำนวณ}
เนื่องจากตัวเกมเสริมทักษะวิชาวิทยาการคำนวณ นั้นแฝงไปด้วยหลายศาสตร์ด้วยกัน เช่น STEM Education, Coding
ซึ่งในอนาคตตัวเกมกสริมทักษะวิชาวิทยาการคำนวณเองนั้นไม่ใช่แค่จะสามารถพัฒนาทักษะการคิดเชิงคำนวณของนักเรียนชั้นประถมศึกษา
แต่ยังสามารถพัฒนาทักษะการคิดเชิงคำนวณของครูได้เช่นกัน หรือรวสมไปถึงการบูรณาการระหว่างหลักสูตร STEM Education และ Coding
\subsection{ด้านระบบของเกม}
ในตัวเกมเสริมทักษะวิชาวิทยาการคำนวณเองนั้นยังสามารถต่อยอดระบบต่างๆ ได้มากมาย ดังนี้
\subsubsection{ระบบค่าเงินภายในเกม}
ในแต่ละด่านนอกจากจะผู้ใช้จะบรรลุเป้าหมายคือการเข้าเส้นชัยแล้วยังมีการเพิ่มความท้าทายอย่างเช่นการเก็บเหรียญภายในเกมเพื่อนำไปซื้อของตลาดภายในเกม 
โดยตลาดภายประกอบไปด้วย
\begin{enumerate}
    \item การซื้อตัวละครภายในเกมเพื่อเปลี่ยนการแสดงผลตัวละครที่ใช้อยู่
    \item การซื้อสิ่งของสวมใส่ตัวละครเพื่อเปลี่ยนการแสดงผลสิ่งของสวมใส่ที่ตัวละครกำลังสวมใส่อยู่
\end{enumerate}

\subsection{Blockly}
ในปัจจุบันผู้คนต่างๆได้ให้ความสำคัญของการคิดเชิงคำนวณเป็นอย่างมาก ซึ่งเครื่องมือที่ใช้พัฒนา Blockly นั้นได้เปิดออกมาเป็น Open source
เลยทำให้มีผู้พัฒนามากมายเข้ามาพัฒนา Blockly ให้มีหน้าตาและระบบที่ดีขึ้น ดังนั้นผู้พัฒนาเลยคิดว่าการที่ไปใช้ Blockly ที่เป็นอันใหม่นั้น
จะเป็นการเพิ่มความสวยงามของรูปลักษณ์และหน้าตาเกมเสริมทักษะวิชาวิทยาการคำนวณเป็นอย่างมาก

\subsection{WebView}
จากการที่ผู้พัฒนาได้ทำสหกิจศึกษาที่บริษัท Logixed จำกัด ทำให้ผู้พัฒนาได้เห็นถึงความสำคัญในการพัฒนาเว็ปไซท์ Blockly ที่มี Framework เข้ามาช่วยในการจัดการ
แต่เนื่องจากปัจจุบันตัวระบบที่ทำงานให้กับ WebView ของเกมเสริมทักษะวิชาวิทยาการคำนวณนั้นเป็น Pure Javascript ซึ่งไม่มี Framework หรือ Node Module ต่างๆ เข้ามาช่วยในการจัดการ
ทำให้ผู้พัฒนาเล็งเห็นถึงความสำคัญและคิดว่าสามารถเปลี่ยนระบบที่ทำงานให้กับ WebView มาเป็นในรูปแบบของการใช้ Framework หรือ Node Module

