\chapter{\ifcpe บทสรุปและข้อเสนอแนะ\else Conclusions and Discussion\fi}

\section{\ifcpe สรุปผล\else Conclusions\fi}
% ทางผู้พัฒนาได้เล็งเห็นถึงความสำคัญของวิชาวิทยาการคำนวณเลยได้ศึกษาวิชาวิทยาการคำนวณซึ่งเป็นวิชาล่าสุดที่ถูกนำมาเพิ่มในรายวิชาวิทยาศาสตร์ในปี พศ. 2560 โดยหลักๆ ในตัวรายวิชาจะเป็นเกี่ยวกับ
% เทคโนโลยี ประกอบไปด้วย การเขียนโปรแกรม, การใช้เทคโนโลยี, การมีภูมิคุ้มกันต่อเทคโนโลยี, การพัฒนาทักษะการคิดอย่างเป็นขั้นเป็นตอน, ฯลฯ
% และเมื่อลงพื้นที่เพื่อไปศึกษาวิชาวิทยาการคำนวณ ณ โรงเรียนบ้านดอยเต่าซึ่งเป็นโรงเรียนรอบนอก ทำให้เราได้เล็งเห็นปัญหาของตัวเด็กนักเรียนเกี่ยวกับวิชาวิทยาการคำนวณ นั่นก็คือ
% การที่ตัวเด็กนักเรียนเองไม่สามามรถเรียนรู้วิชาวิทยาการคำนวณได้เต็มที่ในหลายๆ ปัจจัยที่เกี่ยวข้อง เช่นการที่เด็กไม่ชอบเรียนวิชาวิทยาการคำนวณ
% เพราะว่าไม่มีครูสอนหรือรวมไปถึงตัวเด็กเองไม่ได้ตั้งใจเรียนอยู่แล้ว ทำให้เกิดโครงงานนี้มาเพื่อเป็นสื่อและเครื่องมือผ่านเกมที่เป็น Mobile เกมผ่านโทรศัพท์มือถือเพื่อการพัฒทักษะวิชาวิทยาการคำนวณ
% โดยที่ผู้พัฒนาจะมุ่งเน้นไปที่ทักษะการคิดอย่างเป็นขั้นเป็นตอนซึ่งเป็นทักษะที่สำคัญในการแก้ไขปัญหาต่างๆ
% โครงงานนี้มีจุดประสงค์ดังต่อไปนี้คือ ลดความเหลื่อมล้ำทางด้านการศึกษา, ทำให้เด้กนักเรียนสามารถนำความรู้ที่ได้ไปแก้ไขปัญหาในชีวิตประจำวันได้ และ
% ผลักดันหลุกสูตรวิชาวิทยาการคำนวณ เนื้อหาในตัวเกมที่ผู้พัฒนาสร้างขึ้นจะเน้นไปที่การพัฒนาทักษะการคิดอย่างเป็นขั้นเป็นตอน
% ซึ่งผู้พัฒนาจะอิงตามหลักสูตรวิชาวิทยาการคำนวณของ สสวท. โดยเนื้อหาในตัวเกมนั้นจะเป็นในรูปแบบของการแก้ไขปริศนากล่าวคือ ผู้เล่น(เด็กนักเรียน)
% ต้องแก้ปริศนาคือการให้ตัวละครในเกมเดินไปยังจุดหมายหรือเส้นชัยผ่านการเลือกวาง Blockly Code ซึ่งรูปแบบของด่านในเกม
% จะอิงตามหนังสือวิชาวิทยาการคำนวณของ สสวท. โดยผู้พัฒนาได้แบ่งการทดสอบเป็น 2 อย่าง คือการทดสอบกับกลุ่มตัวอย่างซึ่งก็คือนักเรียนชั้นประถมศึกษาปีที่ 3-6 จำนวน 71 คนเพื่อทดสอบ
% ว่าเครื่องมือหรือเกมเสริมทักษะวิชาวิทยาการคำนวณนั้นสามารถเสริมทักษะวิชาวิทยาการคำนวณได้จริงผ่านการทดสอบ pre-test, post-test โดยข้อสอบจะอิงตามหนังสือวิชาวิทยาการคำนวณของ สสวท. ซึ่งผลที่ได้ออกมาคือนักเรียนทุกคนนั้นมีพัฒนาการหลังจากการ
% เล่นเกมเสริมทักษะวิชาวิทยาการคำนวณ เนื่องจากโครงงานนี้เป็นการสร้างซอฟแวร์ซึ่งเป็นในรูปแบบของเกมทำให้ต้องมีการทดสอบระบบต่างๆ ซึ่งประกอบไปด้วย UX, UI, เนื้อหาในเกม, ลำดับความยากง่ายของด่านต่อไป เป็นต้น ดังนั้น
% ทางผู้พัฒนาเลยมีการทดสอบตัวระบบของเกมเสริมทักษะวิชาวิทยาการคำนวณผ่านผู้เชี่ยวชาญจำนวน 3 ท่านซึ่งมีประสบการทางด้านการเขียนโปรแกรมผ่าน Unity ไม่ต่ำกว่า 3 ปี โดยผู้พัฒนาใช้โมเดล IOC ในการวัดผลและผลที่ออกมา
% คือทุกหัวข้อในการทดสอบผ่านทั้งหมด ดังนั้นจึงสรุปได้ว่าโครงงานเกมเสริมทักษะวิชาวิทยาการคำนวณนั้นสามารถลดความเหลื่อมล้ำทางด้านการศึกษากล่างคือ หลังจากนักเรียนได้เล่นเกมเสริมทักษะวิชาวิทยาการคำนวณไป
% นักเรียนสามารถทำข้อสอบวิชาวิทยาการคำนวณที่อิงจากหนังสือ สสวท. ได้ซึ่งไม่ต่างกับเด็กนักเรียนที่เรียนตามหนังสือวิชาวิทยาการคำนวณของ สสวท. 
% นอกจากนี้นักเรียนเองสามารถนำความรู้ไปแก้ไขปัญหาในชีวิตจริงได้กล่าวคือ ใน post-test ทางผู้พัฒนาได้เพิ่มความลึกของข้อสอบโดย
% เพิ่มการคิดต่อยอดเข้าไปซึ่งตัวเด็กนักเรียนเองก็สามารถทำได้ หลังจากที่ผู้พัฒนาได้เข้าไปลงพื้นที่เพื่อทดสอบเกมเสริมทักษะวิชาวิทยาการคำนวณกับกลุ่มตัวอย่างทำให้ทางโรงเรียนที่ไปลงพื้นที่นั้น
% สนใจในเกมเสริมทักษะวิชาวิทยาการคำนวณเป็นอย่างมากอละได้เล็งเห็นถึงความสำคัญของหลักสูตรวิชาวิทยาการคำนวณซึ่งเป็นการผลักดันหลักสูตรวิชาวิทยาการคำนวณตามที่ผู้พัฒนาได้ตั้งไว้
จากผลการทดสอบจากบทที่ \ref{eval} เกมเสริมทักษะวิชาวิทยาการคำนวณเป็นเกมที่สามารถเพิ่มทักษะการคิดเชิงคำนวณและการคิดอย่างเป็นขั้นเป็นตอนได้ดีเยี่ยม สมควรเป็นอย่างยิ่งที่จะพัฒนาต่อ
ซึ่งบรรลุจุดประสงค์ที่ได้ตั้งไว้ กล่าวคือ การลดความเหลือมล้ำทางด้านการศึกษา โดยหลังจากนักเรียนได้เล่นเกมเสริมทักษะวิชาวิทยาการคำนวณไปแล้ว นักเรียนสามารถทำข้อสอบวิชาวิทยาการคำนวณที่อิงจากหนังสือ สสวท. ได้ไม่ต่างกับเด็กนักเรียนที่เรียนตามหนังสือวิชาวิทยาการคำนวณของ สสวท. 
นอกจากนี้ นักเรียนเองสามารถนำความรู้ไปแก้ไขปัญหาในชีวิตจริงได้ กล่าวคือ ใน post-test ทางผู้พัฒนาได้เพิ่มความลึกของข้อสอบ โดยเพิ่มการคิดต่อยอดเข้าไป ซึ่งตัวเด็กนักเรียนเองก็สามารถทำได้ 

หลังจากที่ผู้พัฒนาได้เข้าไปลงพื้นที่เพื่อทดสอบเกมเสริมทักษะวิชาวิทยาการคำนวณกับกลุ่มตัวอย่าง ทำให้ทางโรงเรียนที่ไปลงพื้นที่นั้นสนใจในเกมเสริมทักษะวิชาวิทยาการคำนวณเป็นอย่างมาก และได้เล็งเห็นถึงความสำคัญของหลักสูตรวิชาวิทยาการคำนวณ ซึ่งเป็นการผลักดันหลักสูตรวิชาดังกล่าว ตามที่ผู้พัฒนาได้ตั้งมั่นไว้

สุดท้ายนี้ โครงงานหนีจากวังวน (Escape) เกมเสริมทักษะวิชาวิทยาการคำนวณนี้ สามารถนำไปใช้ในการพัฒนาทักษะวิชาวิทยาการคำนวณ และการคิดอย่างเป็นขั้นเป็นตอนของเด็กนักเรียนได้อย่างมีประสิทธิภาพ


\section{\ifcpe ปัญหาที่พบและแนวทางการแก้ไข\else Challenges\fi}

ในการทำโครงงานนี้ พบว่าเกิดปัญหาหลักๆ อยู่ 2 ส่วน ดังนี้

\subsection{ปัญหาทางด้านเครื่องมือสร้างเกม (Unity)}
เกมเสริมทักษะวิชาวิทยาการคำนวณที่ได้พัฒนาขึ้นมานั้น เป็นเกมที่ไม่เหมือนเกมทั่วไปที่สร้างด้วย
Unity โดยตัวเกมหลักๆ จะประกอบไปด้วยกันอยู่ 2 ส่วนคือส่วนที่เป็น Unity แสดงผลต่างๆ และหน้าเว็บ ซึ่งทางหน้าเว็บผู้พัฒนาต้องเก็บไว้ในตัว resources file
บน Unity และเมื่อเริ่มการ build โปรเจคที่โครงสร้างไฟล์ต่างจากปกติ ทำให้ Gradle linker ทำงานผิดพลาด ดังนั้น ผู้พัฒนาจึงต้องทำการ manual link ไฟล์ต่างๆ เอง
ซึ่งเป็นส่วนที่ทำให้เสียเวลาในการพัฒนาตัวเกมมาก

\subsection{ปัญหาระหว่างการทดสอบกับกลุ่มตัวอย่าง}
ปัญหาระหว่างการทดสอบ แบ่งได้ออกเป็น 2 หัวข้อ ดังนี้
% \subsubsection{เมื่อทดสอบกับกลุ่มตัวอย่างสิ่งที่เรียกว่าบัคนั้นได้เกิดขึ้นจากการเขียนโปรแกรมของทางผู้พัฒนา เช่น เดิมตกแมพ, ตัวละครบินได้, ตัวแสดงผลหน้าเว็ปนั้นค้าง ซึ่งได้รับการแก้ไข้หลังจากไปทดสอบกับกลุ่มตัวอย่างเสร็จแล้ว}
% \subsubsection{การเข้าไปทดสอบในช่วงที่มีโรคระบาด Covid-19 ทำให้มีอุปสรรคเช่น การเข้าใกล้ชิดกับนักเรียนและการรวมกลุ่มกันของนักเรียน ทำให้ต้องมีขั้นตอนในการปฎิบัติตามมาตรการป้องกันคือ การตรวจหาเชื้อ Covid-19 ก่อนเข้าไปทดสอบกับกลุ่มตัวอย่างและต้องทดสอบกับกลุุ่มตัวอย่างทีละ 2-3 คน}
\begin{enumerate}
    \item เมื่อทดสอบกับกลุ่มตัวอย่าง สิ่งที่เรียกว่าบัคนั้นได้เกิดขึ้นจากการเขียนโปรแกรมของทางผู้พัฒนา เช่น เดินตกแมพ, ตัวละครบินได้, ตัวแสดงผลหน้าเว็บนั้นค้าง ซึ่งได้รับการแก้ไข้หลังจากไปทดสอบกับกลุ่มตัวอย่างเสร็จแล้ว
    \item การเข้าไปทดสอบในช่วงที่มีโรคระบาด Covid-19 ทำให้มีอุปสรรคเช่น การเข้าใกล้ชิดกับนักเรียนและการรวมกลุ่มกันของนักเรียน ทำให้ต้องมีขั้นตอนในการปฎิบัติตามมาตรการป้องกันคือ การตรวจหาเชื้อ Covid-19 ก่อนเข้าไปทดสอบกับกลุ่มตัวอย่างและต้องทดสอบกับกลุุ่มตัวอย่างทีละ 2--3 คน
\end{enumerate}

\section{\ifcpe%
ข้อเสนอแนะและแนวทางการพัฒนาต่อ
\else%
Suggestions and further improvements
\fi
}
ข้อเสนอแนะเพื่อพัฒนาโครงงานนี้ต่อไป มีดังนี้
\subsection{การนำเครื่องมือไปใช้ในการต่อยอดและพัฒนาทักษะทางด้านการคิดเชิงคำนวณ}
เนื่องจากตัวเกมเสริมทักษะวิชาวิทยาการคำนวณนั้นแฝงไปด้วยหลายศาสตร์ด้วยกัน เช่น STEM education, coding
ซึ่งในอนาคต ตัวเกมเสริมทักษะวิชาวิทยาการคำนวณเองนั้นไม่ใช่แค่จะสามารถพัฒนาทักษะการคิดเชิงคำนวณของนักเรียนชั้นประถมศึกษา
แต่ยังสามารถพัฒนาทักษะการคิดเชิงคำนวณของครูได้เช่นกัน หรือรวมไปถึงการบูรณาการระหว่างหลักสูตร STEM education และ coding

\subsection{ด้านระบบของเกม}
ในตัวเกมเสริมทักษะวิชาวิทยาการคำนวณเองนั้นยังสามารถต่อยอดระบบต่างๆ ได้มากมาย ดังนี้

\subsubsection{ระบบค่าเงินภายในเกม}
ในแต่ละด่าน นอกจากจะผู้ใช้จะบรรลุเป้าหมายคือการเข้าเส้นชัยแล้ว ยังมีการเพิ่มความท้าทาย อย่างเช่นการเก็บเหรียญภายในเกมเพื่อนำไปซื้อของตลาดภายในเกม 
โดยตลาดภายในเกมประกอบไปด้วย
\begin{enumerate}
    \item การซื้อตัวละครภายในเกมเพื่อเปลี่ยนการแสดงผลตัวละครที่ใช้อยู่
    \item การซื้อสิ่งของสวมใส่ตัวละครเพื่อเปลี่ยนการแสดงผลสิ่งของสวมใส่ที่ตัวละครกำลังสวมใส่อยู่
\end{enumerate}

\subsection{Blockly}
ในปัจจุบัน ผู้คนต่างๆ ได้ให้ความสำคัญต่อการคิดเชิงคำนวณเป็นอย่างมาก ซึ่งเครื่องมือที่ใช้พัฒนา Blockly นั้นได้เปิดออกมาเป็น open source
จึงทำให้มีผู้พัฒนามากมายเข้ามาพัฒนา Blockly ให้มีหน้าตาและระบบที่ดีขึ้น ดังนั้น ผู้พัฒนาจึงคาดว่าการใช้ Blockly รุ่นใหม่ๆ นั้น
จะเป็นการเพิ่มความสวยงามของรูปลักษณ์และหน้าตาเกมเสริมทักษะวิชาวิทยาการคำนวณเป็นอย่างมาก

\subsection{WebView}
จากการทำสหกิจศึกษาที่บริษัท Logixed จำกัด ทำให้ผู้พัฒนาได้เห็นถึงความสำคัญในการพัฒนาเว็บไซต์ Blockly ที่มี framework เข้ามาช่วยในการจัดการโค้ดต่างๆ ให้เป็นสัดส่วนและไม่ต้องเขียนโค้ดซ้ำๆ โดยมีตัวช่วยเป็น libraries ต่างๆ
แต่เนื่องจากปัจจุบันตัวระบบที่ทำงานให้กับ WebView ของเกมเสริมทักษะวิชาวิทยาการคำนวณนั้นเป็น pure Javascript ซึ่งไม่มี framework หรือ Node modules ต่างๆ เข้ามาช่วยในการจัดการ
ทำให้ผู้พัฒนาเล็งเห็นถึงความสำคัญ และคิดว่าสามารถเปลี่ยนระบบที่ทำงานให้กับ WebView มาเป็นในรูปแบบของการใช้ framework หรือ Node modules

