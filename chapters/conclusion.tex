\chapter{\ifcpe บทสรุปและข้อเสนอแนะ\else Conclusions and Discussion\fi}

\section{\ifcpe สรุปผล\else Conclusions\fi}

ทางผู้พัฒนาได้เล็งเห็นถึงความสำคัญของวิชาวิทยาการคำนวณเลยได้ศึกษาวิชาวิทยาการคำนวณซึ่งเป็นวิชาล่าสุดที่ถูกนำมาเพิ่มในรายวิชาวิทยาศาสตร์ในปี พศ. 2560 โดยหลักๆ ในตัวรายวิชาจะเป็นเกี่ยวกับ
เทคโนโลยี ประกอบไปด้วย การเขียนโปรแกรม, การใช้เทคโนโลยี, การมีภูมิคุ้มกันต่อเทคโนโลยี, การพัฒนาทักษะการคิดอย่างเป็นขั้นเป็นตอน, ฯลฯ
และเมื่อลงพื้นที่เพื่อไปศึกษาวิชาวิทยาการคำนวณ ณ โรงเรียนบ้านดอยเต่าซึ่งเป็นโรงเรียนรอบนอก ทำให้เราได้เล็งเห็นปัญหาของตัวเด็กนักเรียนเกี่ยวกับวิชาวิทยาการคำนวณ นั่นก็คือ
การที่ตัวเด็กนักเรียนเองไม่สามามรถเรียนรู้วิชาวิทยาการคำนวณได้เต็มที่ในหลายๆ ปัจจัยที่เกี่ยวข้อง เช่นการที่เด็กไม่ชอบเรียนวิชาวิทยาการคำนวณ
เพราะว่าไม่มีครูสอนหรือรวมไปถึงตัวเด็กเองไม่ได้ตั้งใจเรียนอยู่แล้ว ทำให้เกิดโครงงานนี้มาเพื่อเป็นสื่อและเครื่องมือผ่านเกมที่เป็น Mobile เกมผ่านโทรศัพท์มือถือเพื่อการพัฒทักษะวิชาวิทยาการคำนวณ
โดยที่ผู้พัฒนาจะมุ่งเน้นไปที่ทักษะการคิดอย่างเป็นขั้นเป็นตอนซึ่งเป็นทักษะที่สำคัญในการแก้ไขปัญหาต่างๆ
โครงงานนี้มีจุดประสงค์ดังต่อไปนี้คือ ลดความเหลื่อมล้ำทางด้านการศึกษา, ทำให้เด้กนักเรียนสามารถนำความรู้ที่ได้ไปแก้ไขปัญหาในชีวิตประจำวันได้ และ
ผลักดันหลุกสูตรวิชาวิทยาการคำนวณ เนื้อหาในตัวเกมที่ผู้พัฒนาสร้างขึ้นจะเน้นไปที่การพัฒนาทักษะการคิดอย่างเป็นขั้นเป็นตอน
ซึ่งผู้พัฒนาจะอิงตามหลักสูตรวิชาวิทยาการคำนวณของ สสวท. โดยเนื้อหาในตัวเกมนั้นจะเป็นในรูปแบบของการแก้ไขปริศนากล่าวคือ ผู้เล่น(เด็กนักเรียน)
ต้องแก้ปริศนาคือการให้ตัวละครในเกมเดินไปยังจุดหมายหรือเส้นชัยผ่านการเลือกวาง Blockly Code ซึ่งรูปแบบของด่านในเกม
จะอิงตามหนังสือวิชาวิทยาการคำนวณของ สสวท. {not done yet.}

\section{\ifcpe ปัญหาที่พบและแนวทางการแก้ไข\else Challenges\fi}

ในการทำโครงงานนี้ พบว่าเกิดปัญหาหลักๆ ดังนี้

\section{\ifcpe%
ข้อเสนอแนะและแนวทางการพัฒนาต่อ
\else%
Suggestions and further improvements
\fi
}

ข้อเสนอแนะเพื่อพัฒนาโครงงานนี้ต่อไป มีดังนี้
