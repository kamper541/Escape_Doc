\chapter{\ifproject%
\ifcpe การทดลองและผลลัพธ์\else Experimentation and Results\fi
\else%
\ifcpe การประเมินระบบ\else System Evaluation\fi
\fi}

% ในบทนี้จะทดสอบเกี่ยวกับการทำงานในฟังก์ชันหลักๆ
% การทดสอบระบบของเราจะทดสอบโดยใช้ User Test ซึ่งเราจะนำไปให้กลุ่มทดลองที่เป็นเด็กนักเรียนชั้นประถมศึกษาปีที่ 4-6 โรงเรียนรอบนอก
% ชั้นปีละ 20 คนในการทดสอบโดย เราจะวัดผลโดยการใช้ pre-test เพื่อวัดระดับของเด็กนักเรียนก่อนเล่นเกมและ post-test เพื่อวัดระดับของเด็กนักเรียนหลังเล่นเกม
% ซึ่งในเนื้อหา pre-test และ post-test จะเป็นแบบฝึกหัดเนื้อหาวิชาวิทยาการคำนวณ โดยเราจะนำผล pre-test และ post-test มาเทียบกันเพื่อ
% ประเมินตัวเกมของเราว่าจะสามารถเป็นสื่อการเรียนการสอนให้เด็กนักเรียนได้เข้าใจในวิชาวิทยาการคำนวณมากขึ้นจริงไหม 
% รวมไปถึงภายในตัวเกมเองก็จะมีการเก็บ Score ของผู้เล่นโดยแต่ละด่านจะมีจำนวณจำกัดในการใช้ Block Code ซึ่งถ้านักเรียนใช้จำนวณคำสั่งเกินมาจากที่กำหนดไว้คะแนนของนักเรียนจะ
% ลดลงตามจำนวณคำสั่งที่เพิ่มมาและจะมีการเก็บเวลาที่ใช้ในแต่ละด่านของนักเรียนแต่ละคนเดียว ทั้งหมดนี้จะนำไปขึ้น Score Board เพิ่งแสดงให้เห็นว่าตัวนักเรียนเองได้คะแนนจากด่านนี้ๆ เท่าไหร่
% \CIreply{ใจความของย่อหน้านี้คืออะไร
การประเมินระบบของโครงการนี้ จะมีการประเมิณอยู่ 2 วิธีคือ User Test ซึ่งจะเป็นเด็กนักเรียนชั้นประถมศึกษาปีที่ 3-6 โดยผ่านเครื่องมือการ Pre-Test
Post-Test และอีกวิธีคือ Expert Test เป็นการประเมินประเมินระบบภายในเกม ความยากง่ายของเกม (Game Design) และ UX/UI ผ่าน
เครื่องมือที่เรียกว่า IOC

\section{User Test}
การศึกษาเรื่องการวัดผลวิชาวิทยาการคำนวณสำหรับนักเรียนชั้นประถมศึกษาปีที่ 3 – 6 
โรงเรียนบ้านออนใต้และโรงเรียนบ้านโฮ้ง อำเภอสันกำแพง
 จังหวัดเชียงใหม่ โดยใช้การวัดผลแบบ Pre-test Post-test ก่อนและหลังเล่นเกม ผู้ศึกษานำเสนอผลวิเคราะห์ข้อมูลตามลำดับดังนี้
\subsection{Pre-test Post-test}
ผลสัมฤทธิ์ทางการเรียนรู้เรื่องวิชาวิทยาการคำนวณหมวดการคิดอย่างเป็นขั้นเป็นตอบโดยใช้เกมเสริมทักษะวิชาวิทยาการคำนวณ (Escape) ของนักเรียนชั้นประถมศึกษาปีที่ 3 – 6
โรงเรียน บ้านออนใต้และโรงเรียนบ้านโฮ้ง อำเภอสันกำแพง จังหวัดเชียงใหม่\par
ได้สร้างเกมเสริมทักวิชาวิทยาการคำนวณสำหรับนักเรียนชั้นประถมศึกษาปีที่ 3 – 6 โรงเรียน
บ้านออนใต้และโรงเรียนบ้านโฮ้ง อำเภอสันกำแพง จังหวัดเชียงใหม่ จำนวน 1 เกม
โดยภายในเกมประกอบไปด้วยด่านจำนวน 30 ด่าน
แต่ละด่านประกอบไปด้วยรายละเอียดดังต่อไปนี้\par
\GBreply{ใส่เป็นภาพได้ไหม?}
\begin{center}
    \begin{tabular}{|c | m{35em}|} 
     \hline
     ด่านที่ & เนื้อหา\\ [0.5ex] 
     \hline\hline
     1 &  ศึกษาวิธีการเล่นเกมโดยให้ตัวละครเดินไป 1 ก้าว \\ 
     \hline
     2 &  ศึกษาวิธีการเปลี่ยนค่าในกล่องคำสั่งโดยให้ตัวละครเดินไป 2 ก้าว \\ 
     \hline
     3 &  ศึกษาวิธีการเปลี่ยนค่าในกล่องคำสั่งโดยให้ตัวละครเดินไป 4 ก้าว \\ 
     \hline
     4 &  ศึกษาคำสั่งใหม่ (หมุน) โดยให้ตัวละครหมุน 1 ครั้งและเดิน 1 ก้าว \\ 
     \hline
     5 &  ศึกษาคำสั่งใหม่ (หมุน) โดยให้ตัวละครหมุน 1 ครั้งและเดิน 4 ก้าว \\ 
     \hline
     6 &  ผสมผสานคำสั่งโดยให้ตัวละครเดินและหมุน \\ 
     \hline
     7 &  ผสมผสานคำสั่งโดยให้ตัวละครเดินและหมุนไปในอีกทิศทาง \\ 
     \hline
     8 &  เพิ่มระดับความยากจากด่านก่อนหน้าโดยต้องใช้คำสั่ง หมุนและเดิน มากขึ้น \\ 
     \hline
     9 &  เพิ่มระดับความยากจากด่านก่อนหน้าโดยต้องใช้คำสั่ง หมุนและเดิน มากขึ้น \\ 
     \hline
     10 &  เพิ่มระดับความยากจากด่านก่อนหน้าโดยต้องใช้คำสั่ง หมุนและเดิน มากขึ้น \\ 
     \hline
     11 &  ศึกษาคำสั่งใหม่ (ทำซ้ำ) โดยให้ผู้ใช้เดินไป 1 ด้าวและครอบด้วยคำสั่งทำซ้ำไ \\ 
     \hline
     12 &  ผสมผสานคำสั่ง เดิน,หมุน และทำซ้ำโดยให้ตัวละครเดินเป็นตัว U คว่ำ \\ 
     \hline
     13 &  ผสมผสานคำสั่ง เดิน,หมุน และทำซ้ำโดยให้ตัวละครเดินเป็นซิกเซ็ก \\ 
     \hline
     14 &  ผสมผสานคำสั่ง เดิน,หมุน และทำซ้ำโดยให้ตัวละครเดินเป็นตัว U คว่ำที่ใหญ่กว่า \\ 
     \hline
     15 &  ผสมผสานคำสั่ง เดิน,หมุน และทำซ้ำโดยให้ตัวละครเดินเป็นตัว U คว่ำ โดยมีการหลอก \\ 
     \hline
     16 &  ศึกษาคำสั่งใหม่ (ตัวแปร) \\ 
     \hline
     17 &  ศึกษาวิธีเปลี่ยนค่าของคำสั่งตัวแปร \\ 
     \hline
     18 &  ผสมผสานคำสั่ง เดิน,ตัวแปร และหมุน \\ 
     \hline
     19 &  ผสมผสานคำสั่ง เดิน,ตัวแปร,หมุน และทำซ้ำโดยให้ตัวละครเดินเป็นรูปเปลือกห้อง \\ 
     \hline
     20 &  ผสมผสานคำสั่ง เดิน,ตัวแปร,หมุน และทำซ้ำโดยให้ตัวละครเดินเป็นรูปเปลือกห้องที่ใหญ่ขึ้น \\ 
     \hline
     21 &  ศึกษาคำสั่งใหม่ (เหยียบบน Block สี) โดยให้เดินไปข้างหน้า \\ 
     \hline
     22 &  ผสมผสานคำสั่ง เดิน, หมุน และเหยียบบน Block สี \\ 
     \hline
     23 &  ผสมผสานคำสั่ง เดิน, หมุน และเหยียบบน Block สีโดยมีการเปลี่ยนค่าของคำสั่ง \\ 
     \hline
     24 &  ผสมผสานคำสั่ง เดิน, หมุน, เหยียบบน Block สี และ ทำซ้ำ \\ 
     \hline
     25 &  ผสมผสานคำสั่ง เดิน, หมุน, เหยียบบน Block สี และ ทำซ้ำ โดยให้เดินเป็นรูปตัว U คว่ำ \\ 
     \hline
     26 &  เพิ่มจำนวน Block สีที่เหยียบได้ โดยในด่านจะมี 2 สี \\ 
     \hline
     27 &  ผสมผสานคำสั่ง เดิน, หมุน, เหยียบบน Block สี และ ทำซ้ำ โดยให้เดินเป็นซิกเซ๊ก \\ 
     \hline
     28 &  ผสมผสานคำสั่ง เดิน, หมุน, เหยียบบน Block สี และ ทำซ้ำ โดยให้เดินเป็นซิกเซ๊กโดยเพิ่มจำนวนคำสั่งที่ใช้ \\ 
     \hline
     29 &  ผสมผสานคำสั่ง เดิน, หมุน, เหยียบบน Block สี และ ทำซ้ำ โดยให้เดินเป็นรูปตัว U คว่ำพร้อมทั้งมีการหลอกการใช้คำสั่ง \\ 
     \hline
     30 &  ผสมผสานคำสั่ง เดิน, หมุน, เหยียบบน Block สี และ ทำซ้ำ โดยมี Block สีที่เหยียบได้ 3 สี \\ 
     \hline
    \end{tabular}
\end{center}

จากผลการประเมินความเหมาะสมของเกมเสริมทักษะวิชาวิทยาการคำนวณ (Escape) สำหรับนักเรียนชั้นประถมศึกษาปีที่ 
3 – 6 โรงเรียน บ้านออนใต้และโรงเรียนบ้านโฮ้ง อำเภอสันกำแพง จังหวัดเชียงใหม่ 
โดยมีผู้เชี่ยวชาญจำนวน 2 ท่านผ่านการพูดคุยและปรึกษา ปรากฏผลดังนี้ ได้ปรับแก้เนื้อหาที่อิงตามหลักสูตรวิชาวิทยาการคำนวณภายในด่าน 
ให้เหมาะสมกับศักยภาพและวัยของผู้เรียน และเมื่อนำเกมเสริมทักษะวิชาวิทยาการคำนวณ (Escape) 
ไปใช้กับกลุ่มเป้าหมายพบประเด็นที่น่าสนใจดังนี้

\begin{enumerate}
    \item นักเรียนมีความกระตือรือร้นในการทำกิจกรรมโดยใช้เกมเสริมทักษะวิชาวิทยาการคำนวณ (Escape) และได้รับสนุกสนานควบคู่ไปกับความรู้ 
    \item เสริมสร้างกระบวนการเรียนรู้ทางเทคโนโลยีและฝึกให้เด็กได้ใช้ความคิดอย่างเป็นขั้นเป็นตอนตามหลักสูตรแกนกลางการศึกษาขั้นพื้นฐานพุทธศักราช 2560 ในรายวิชาวิทยาศาสตร์ (วิทยาการคำนวณ)
\end{enumerate}
ผลการเรียนรู้เรื่องวิชาวิทยาการคำนวณสำหรับนักเรียนชั้นประถมศึกษาปีที่ 3 – 6 โรงเรียน บ้านออนใต้และโรงเรียนบ้านโฮ้ง อำเภอสันกำแพง จังหวัดเชียงใหม่\par
คะแนนจากการทำแบบทดสอบวัดผลการเรียนรู้ของวิชาวิทยาการคำนวณหมวดหมู่การคิดอย่างเป็นขั้นเป็นตอน โดยใช้เครื่องมือคือเกมเสริมทักษะวิชาวิทยาการคำนวณ (Escape)\par
\bigskip

\begin{tikzpicture}

    \begin{axis} [
        xtick={ป.3,ป.4,ป.5,ป.6},  % Use this to decide which tickmarks to print
        xticklabel style={text height=2ex}, % This aligns all letters on the same line, if it is missing, 'a' and 'b' are at different heights
        ymin=0,
        ylabel=คะแนน,
        ymin=0,
        symbolic x coords={ป.3,ป.4,ป.5,ป.6},
        legend style={at={(0.5,-0.1)},
        anchor=north,legend columns=-5},
        width=11cm,
        ybar,
        bar width=7pt,
    ]
     
    \addplot[red!20!black,fill=red!80!white,ybar,fill] coordinates {(ป.3,34) (ป.4,38) (ป.5,67) (ป.6,73)};
    \addplot[blue!20!black,fill=blue!80!white,ybar,fill] coordinates {(ป.3,32) (ป.4,40) (ป.5,84) (ป.6,100)};

    \legend {Pre-Test, Post-Test};
     
    \end{axis}
     
\end{tikzpicture}\bigskip

จากการทดสอบนักเรียนชั้นประถมศึกษาชั้นปีที่ 3 - 6 ด้วยข้อสอบวิชาวิทยาการคำนวณก่อนและหลังเล่นเกม นักเรียนทุกคนมีพัฒนาการเพิ่มขึ้นเป็นที่น่าพึงพอใจเมื่อทำข้อสอบที่มีความยากมากขึ้น


\section{Expert Evaluation}
การประเมินในด้านระบบของเกมโดยใช้ผู้เชี่ยวชาญในการประเมินคุณภาพอุปกรณ์(เกมเสริมทักษะวิชาวิทยาการคำนวน) จำนวน 3 ท่าน มีหัวข้อการประเมินประกอบไปด้วย UX/UI ของเกม, ความยากง่ายของเกมเมื่อเทียบกับผู้ใช้ซึ่งก็คือนักเรียนชั้นประถม
ศึกษาปีที่ 3 - 6, ผลที่คาดว่าจะสามารถเพิ่มทักษะกาาคิดอย่างเป็นขั้นเป็นตอนให้กับเด็ก และระบบของเกมโดยรวมประกอบไปด้วย 
การลื่นไหลของ animation, ความเร็วการตอบสอนของเกม (ไม่มี delay), ไม่เจอ error ของระบบ (เด้งหรือค้าง)
\subsection{IOC : Index of item objective congruence}
ผู้พัฒนาได้ใช้โมดเดล IOC ในการทดสอบวัดคุณภาพของเครื่องมือ(เกมเสริมทักษะวิชาวิทยาการคำนวน) ซึ่งผู้เชี่ยวชาญจะ
ประเมินแต่ละหัวข้อด้วย 3 ตัวเลือกซึ่งก็คือ ผ่าน, ไม่ทราบ, ไม่ผ่าน ซึ่งการที่ผู้เชี่ยวชาญจะให้ผ่านหรือไม่นั้นเป็นการตีความเองของผู้เชี่ยวชาญ
โดยที่ในแต่ละหัวข้อจะสามารถผ่านได้เมื่อค่าเฉลี่ยในการเลือกให้ผ่านเป็นค่า 66.7\% กล่าวคือในแต่ละหัวข้อต้องถูกประเมินให้ผ่าน 2 ใน 3 ของจำนวนผู้เชี่ยวชาญ โดยมีผลลัพธ์ดังนี้
\begin{center}
    \begin{tabular}{ |p{3cm}|p{3cm}|p{3cm}|p{3cm}| }
        \hline
        หัวข้อการสอบถาม & ผู้เชี่ยวชาญคนที่ 1 & ผู้เชี่ยวชาญคนที่ 2 & ผู้เชี่ยวชาญคนที่ 3\\
        \hline
        ความเหมาะสมทางด้าน UX & ผ่าน & ผ่าน & ผ่าน\\
        \hline
        ความเหมาะสมทางด้าน UI & ผ่าน & ไม่ทราบ & ผ่าน\\
        \hline
        นักเรียนชั้นประถมศึกษาชั้นปีที่ 3 - 6 สามารถเล่นเกมนี้ได้ & ผ่าน (ได้) & ผ่าน (ได้) & ผ่าน (ได้)\\
        \hline
        ตัวเกมสามารถช่วยเพิ่มทักษะการคิดอย่างเป็นขั้นเป็นตอนได้ & ผ่าน (ได้) & ผ่าน (ได้) & ผ่าน (ได้)\\
        \hline
        ตัวระบบโดยรวมของเกมการลื่นไหนของ animation, ความเร็วในการตอบสนอง (ไม่ delay), ไม่เจอ error ของระบบ (เด้ง/ค้าง) & ผ่าน & ผ่าน & ไม่ทราบ\\
        \hline
    \end{tabular}
\end{center}
การวัดคุณภาพของเครื่องมือ(เกมเสริมทักษะวิชาวิทยาการคำนวน) สรุปได้ว่าทุกหัวข้อนั้นได้ผ่านเกณฑ์ของเครื่องมือวัด IOC ทั้งหมดและเป็นที่น่าพึงพอใจสำหรับผู้พัฒนาโดยที่สามารถ
เป็นที่รองรับได้ว่าตัวเครื่องมือนี้สามารถนำไปใช้ในการพัฒนาทักษะวิชาวิทยาการคำนวณการคิดอย่างเป็นขั้นเป็นตอนของเด็กนักเรียนได้อย่างมีประสิทธิภาพ


