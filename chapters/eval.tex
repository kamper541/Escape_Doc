\chapter{\ifproject%
\ifcpe การทดลองและผลลัพธ์\else Experimentation and Results\fi
\else%
\ifcpe การประเมินระบบ\else System Evaluation\fi
\fi}

% ในบทนี้จะทดสอบเกี่ยวกับการทำงานในฟังก์ชันหลักๆ
การทดสอบระบบของเราจะทดสอบโดยใช้ User Test ซึ่งเราจะนำไปให้กลุ่มทดลองที่เป็นเด็กนักเรียนชั้นประถมศึกษาปีที่ 4-6 โรงเรียนรอบนอก
ชั้นปีละ 20 คนในการทดสอบโดย เราจะวัดผลโดยการใช้ pre-test เพื่อวัดระดับของเด็กนักเรียนก่อนเล่นเกมและ post-test เพื่อวัดระดับของเด็กนักเรียนหลังเล่นเกม
ซึ่งในเนื้อหา pre-test และ post-test จะเป็นแบบฝึกหัดเนื้อหาวิชาวิทยาการคำนวณ โดยเราจะนำผล pre-test และ post-test มาเทียบกันเพื่อ
ประเมินตัวเกมของเราว่าจะสามารถเป็นสื่อการเรียนการสอนให้เด็กนักเรียนได้เข้าใจในวิชาวิทยาการคำนวณมากขึ้นจริงไหม 
% รวมไปถึงภายในตัวเกมเองก็จะมีการเก็บ Score ของผู้เล่นโดยแต่ละด่านจะมีจำนวณจำกัดในการใช้ Block Code ซึ่งถ้านักเรียนใช้จำนวณคำสั่งเกินมาจากที่กำหนดไว้คะแนนของนักเรียนจะ
% ลดลงตามจำนวณคำสั่งที่เพิ่มมาและจะมีการเก็บเวลาที่ใช้ในแต่ละด่านของนักเรียนแต่ละคนเดียว ทั้งหมดนี้จะนำไปขึ้น Score Board เพิ่งแสดงให้เห็นว่าตัวนักเรียนเองได้คะแนนจากด่านนี้ๆ เท่าไหร่
% \CIreply{ใจความของย่อหน้านี้คืออะไร}
