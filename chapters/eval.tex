\chapter{\ifproject%
\ifcpe การทดลองและผลลัพธ์\else Experimentation and Results\fi
\else%
\ifcpe การประเมินระบบ\else System Evaluation\fi
\fi}

% ในบทนี้จะทดสอบเกี่ยวกับการทำงานในฟังก์ชันหลักๆ
% การทดสอบระบบของเราจะทดสอบโดยใช้ User Test ซึ่งเราจะนำไปให้กลุ่มทดลองที่เป็นเด็กนักเรียนชั้นประถมศึกษาปีที่ 4-6 โรงเรียนรอบนอก
% ชั้นปีละ 20 คนในการทดสอบโดย เราจะวัดผลโดยการใช้ pre-test เพื่อวัดระดับของเด็กนักเรียนก่อนเล่นเกมและ post-test เพื่อวัดระดับของเด็กนักเรียนหลังเล่นเกม
% ซึ่งในเนื้อหา pre-test และ post-test จะเป็นแบบฝึกหัดเนื้อหาวิชาวิทยาการคำนวณ โดยเราจะนำผล pre-test และ post-test มาเทียบกันเพื่อ
% ประเมินตัวเกมของเราว่าจะสามารถเป็นสื่อการเรียนการสอนให้เด็กนักเรียนได้เข้าใจในวิชาวิทยาการคำนวณมากขึ้นจริงไหม 
% รวมไปถึงภายในตัวเกมเองก็จะมีการเก็บ Score ของผู้เล่นโดยแต่ละด่านจะมีจำนวณจำกัดในการใช้ Block Code ซึ่งถ้านักเรียนใช้จำนวณคำสั่งเกินมาจากที่กำหนดไว้คะแนนของนักเรียนจะ
% ลดลงตามจำนวณคำสั่งที่เพิ่มมาและจะมีการเก็บเวลาที่ใช้ในแต่ละด่านของนักเรียนแต่ละคนเดียว ทั้งหมดนี้จะนำไปขึ้น Score Board เพิ่งแสดงให้เห็นว่าตัวนักเรียนเองได้คะแนนจากด่านนี้ๆ เท่าไหร่
% \CIreply{ใจความของย่อหน้านี้คืออะไร
การประเมิณระบบของโครงการนี้ จะมีการประเมิณอยู่ 3 วิธีคือ แบบ User Test ความเหมาะสมของเนื้อหาในด่าน
และ ความเหมาะสมของเนื้อหาในเกม

\section{User Test}
\subsection{Pre-test Post-test}
ในส่วนนี้เป็นการประเมินระบบโดยวัดผลลัพธ์จากผู้ใช้ในที่นี่ผู้ใช้ที่ถูกวัดผลคือ เด็กนักเรียนชั้นประถมศึกษาปีที่ 4-5
โดยการใช้ประเมินผลผ่าน pre-test และ post-test ขั้นตอนการวัดผลมีดังนี้
\begin{enumerate}
    \item ให้นักเรียนชั้นประถมศึกษาปีที่ 4-5 ทำแบบฝึกหัด pre-test โดยที่เนื้อหาอิงตามแบบฝึกหัดท้ายบทของวิชาวิทยาการคำนวณ
    \item บันทึกผล pre-test ของนักเรียน
    \item ให้นักเรียนชั้นประถมศึกษาปีที่ 4-5 ได้ทดลองเล่นตัวเกม Escape ที่ผู้พัฒนาเป็นผู้จัดทำ
    \item ให้นักเรียนชั้นประถมศึกษาปีที่ 4-5 ทำแบบฝึกหัด post-test ที่เนื้อหาเหมือนกับ pre-test
    \item บันทึกผล post-test ของนักเรียน
    \item นำผลของ pre-test และ post-test มาเทียบกับเพื่อประเมินผล
\end{enumerate}
\subsection{ประเมินผลตามความพึงพอใจของนักเรียน}
ในส่วนนี้จะเป็นการเก็บข้อความคิดหลังจะที่นักเรียนได้ทดลองเล่นเกม Escape โดยให้นักเรียนกรอกความพึงพอใจไปใน
Google Form โดยจะแบ่งเป็นหัวข้อการประเมินดังนี้ โดยแบ่งเป็น 5 ระดับคือ ไม่เลย น้อย ปานกลาง มาก และมากที่สุด
\begin{itemize}
    \item ความง่ายในการเล่นโดยรวม
    \item ความเข้าใจใน Tutorial
    \item ความลื่นในการขยับของตัวละครในเกม
    \item ความสวยงามของด่าน
    \item ความสวยงามของ UI
    \item ความซับซ้อนของ UI
\end{itemize}
\section{ความเหมาะสมของเนื้อหาในด่าน}
ด้านความเหมาะสมของเนื้อหาในด่านจะวัดผลจากเวลาที่ใช้ในแต่ละด่านไม่ควรจะก้าวกระโดด
เช่น ด่านที่หนึ่งใช้เวลาทำ 5 วิ พอไปด่านที่ 2 ใช้เวลาทำ 30 วิ เป็นต้น
จุดประสงค์ของการประเมินนี้จะทำให้ความยากของด่านไม่กระโดดเกินไปเพื่อให้นักเรียนไม่รู้สึกท้อแท้
ในการเล่นเกม
\section{ความเหมาะสมของเนื้อหาในเกม}
บุคคลภายนอกสามารถสร้างด่านของตัวเองได้โดยการ Input มาเป็นไฟล์ text และระบบจะเป็น
คนจัดการไฟล์เพื่อนสร้างด่านขึ้นมาเอง


