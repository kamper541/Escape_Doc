\maketitle
\makesignature

\ifproject
\begin{abstractTH}

วิทยาการคำนวณ (computing science) เป็นวิชาที่จะเข้ามาแทนวิชาคอมพิวเตอร์หรือ วิชาด้านเทคโนโลยีในที่สอนอยู่ในปัจจุบัน 
รายละเอียดของวิชาวิทยาการคำนวณ ไม่ใช่แค่ให้ผู้เรียน เรียนแค่การเขียนโปรแกรมคอมพิวเตอร์ หรือ เรียนรู้เกี่ยวกับการใช้คอมพิวเตอร์แค่ขั้นพื้นฐานเท่านั้น 
แต่วิชานี้ยังสอนให้เด็กๆ มีกระบวนการคิดเชิงวิเคราะห์อย่างเป็นระบบ และสามารถนำมาปรับใช้เพื่อแก้ไขปัญหาได้อย่างสร้างสรรค์ 
ซึ่งในรายวิชานี้ได้มีการกำหนดขอบเขตการเรียนการสอนเอาไว้ 3 องค์ความรู้ ได้แก่
\begin{enumerate}
\item การคิดเชิงคำนวณ (computational thinking)
\item พื้นฐานความรู้ด้านเทคโนโลยีดิจิทัล (digital technology) 
\item พื้นฐานการรู้เท่าทันสื่อและข่าวสาร (media and information literacy) 
\end{enumerate}
โดยปัญหาที่ทางผู้พัฒนาได้ไปสำรวจมาในโรงเรียนพื้นที่รอบนอกชั้นประถมศึกษา คือ ทางโรงเรียนไม่มีศักยภาพพอที่จะสอนวิชาวิทยาการคำนวณ 
เพราะบุคลากรไม่พร้อม จบไม่ตรงสาย ทำให้เด็กนักเรียนไม่ได้เรียนวิชาวิทยาการคำนวณอย่างที่ควรเป็นในหลักสูตรของกระทรวงศึกษาธิการ 
ซึ่งทางผู้พัฒนาได้เล็งเห็นว่า การที่จะนำเสนอรูปแบบการเรียนการสอนในรูปแบบที่เด็กสามารถเรียนรู้ได้ด้วยตัวเอง และสนุกไปกับมัน 
จึงทำให้เกิดโครงงานเกมเสริมทักษะวิทยาการคำนวณ ขึ้นมา

โครงงานหนีออกจากวังวน (Escape) เป็นเกมเสริมทักษะวิชาวิทยาการคำนวณ ซึ่งจะเน้นด้านการคิดเชิงคำนวณอย่างเป็นขั้นเป็นตอน
สามารถช่วยแก้ไขปัญหาการที่นักเรียนไม่ได้เรียนวิชาวิทยาการคำนวณอย่างที่ควรจะเป็น 
โดยทางผู้พัฒนาจะนำ Blockly เข้ามาใช้ในตัวเกมเพื่ออิงตามหลักสูตรวิชาวิทยาการคำนวณของกระทรวงศึกษาธิการ 
ที่ได้มีการเรียนการสอนโดยใช้ Scratch ที่เป็น Blockly เช่นกัน 
โดยตัวเกมจะเป็นการที่ผู้เล่น (นักเรียน) นำ Blockly มาวางเพื่อบังคับตัวละครในเกมเพื่อแก้ไขปัญหาในแต่ละด่านที่ได้ออกแบบอิงตามแบบฝึกหัดหนังสือของ 
สสวท. วิชาวิทยาการคำนวณ
% \enskip ภาควิชาวิศวกรรมคอมพิวเตอร์จึงได้จัดทำต้นแบบรูปเล่มรายงานโดยใช้ระบบจัดเตรียมเอกสาร
% \LaTeX{} เพื่อช่วยให้นักศึกษาเขียนรายงานได้อย่างสะดวกและรวดเร็วมากยิ่งขึ้น
\end{abstractTH}

\begin{abstract}

Computing science is a subject that will replace Computer or Technology subjects currently offered in schools.
Computing science is not only about programming or the basics of computers, but is also about systematic, analytical thinking skills and creative problem solving.
The scope of the subject is defined into 3 parts
\begin{enumerate}
    \item Computational thinking
    \item Digital technology
    \item Media and information literacy
\end{enumerate}
Our observations on primary schools located in outer areas of Chiang Mai revealed that most of the schools do not have employees specialized in computer science field, resulting in less potential to teach computing science subject to the students, albeit required by the Ministry of Education's curriculum. This causes the students to learn less on the subject. To solve this problem, we came up with this project---\emph{Escape}---to offer a learning tool that students can use on their own and have fun with it.

Escape focuses on enhancing computational thinking skills.
As the Ministry of Education's curriculum use Scratch for the concept of block coding, we use \emph{Blockly}, a similar block-based coding language. In our project, the player (student) can control the character in the game to solve each level by placing blocks of code containing commands that tell the character to move as instructed. Each level is designed according to the exercises in the textbooks written by the
Institute for the Promotion of Teaching Science and Technology (IPST).

% Make sure your abstract sits inside the \texttt{abstract} environment.
\end{abstract}

\iffalse
\begin{dedication}
This document is dedicated to all Chiang Mai University students.

Dedication page is optional.
\end{dedication}
\fi % \iffalse

\begin{acknowledgments}
โครงการ หนีออกจากวังวน (Escape) 
นี้จะประสบความสำเร็จไม่ได้หากไม่ได้รับการสนับสนุนจากคณะวิศวกรรมศาสตร์ มหาวิทยาลัยเชียงใหม่ 
ในด้านการอำนวยความสะดวกและสถานที่ในการพัฒนาโปรแกรม
อาจารย์ชินวัตร อิศราดิสัยกุล ที่ให้คำปรึกษาในการพัฒนาโปรแกรม
รวมถึงพนักงานบริษัท Logixed จำกัด ที่ช่วยให้แบบอย่างและแนวทางของการทำงานที่ดี
และทีมผู้ร่วมพัฒนาที่คอยช่วยเหลือกันในการพัฒนาโปรแกรม 
นอกจากนี้ทางคณะผู้พัฒนายังได้รับกำลังใจจากบิดามารดา พี่น้อง อาจารย์ และ
เพื่อนของผู้พัฒนา ที่อำนวยความสะดวกและให้คำปรึกษาด้านโปรแกรม 
% \texttt{acknowledgment} environment.

\acksign{2022}{3}{17}
\end{acknowledgments}%
\fi % \ifproject

\contentspage

\ifproject
\figurelistpage

\tablelistpage
\fi % \ifproject

% \abbrlist % this page is optional

% \symlist % this page is optional

% \preface % this section is optional
