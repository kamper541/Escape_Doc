\maketitle
\makesignature

\ifproject
\begin{abstractTH}

$\>$วิทยาการคำนวณ (computing science) เป็นวิชาที่จะเข้ามาแทนวิชาคอมพิวเตอร์หรือ วิชาด้านเทคโนโลยีในที่สอนอยู่ในปัจจุบัน 
รายละเอียดของวิชาวิทยาการคำนวณ ไม่ใช่แค่ให้ผู้เรียน เรียนแค่การเขียนโปรแกรมคอมพิวเตอร์ หรือ เรียนรู้เกี่ยวกับการใช้คอมพิวเตอร์แค่ขั้นพื้นฐานเท่านั้น 
แต่วิชานี้ยังสอนให้เด็ก ๆ มีกระบวนการคิดเชิงวิเคราะห์อย่างเป็นระบบและสามารถนำมาปรับใช้เพื่อแก้ไขปัญหาได้อย่างสร้างสรรค์ 
ซึ่งในรายวิชานี้ได้มีการกำหนดขอบเขตการเรียนการสอนเอาไว้ 3 องค์ความรู้ ได้แก่ 1.การคิดเชิงคำนวณ 
(computational thinking) 2.พื้นฐานความรู้ด้านเทคโนโลยีดิจิทัล (digital technology) 
3.พื้นฐานการรู้เท่าทันสื่อและข่าวสาร (media and information literacy) 
โดยปัญหาที่ทางผู้พัฒนาได้ไปสำรวจมาในโรงเรียนพื้นที่รอบนอกชั้นประถมศึกษา คือ ทางโรงเรียนไม่มีศักยภาพพอที่จะสอนวิชาวิทยาการคำนวณ 
เพราะบุคลากรไม่พร้อม จบไม่ตรงสาย ทำให้เด็กนักเรียนไม่ได้เรียนวิชาวิทยาการคำนวณอย่างที่ควรเป็นในหลักสูตรของกระทรวงศึกษาธิการ 
ซึ่งทางผู้พัฒนาได้เล็งเห็นว่า การที่จะนำเสนอรูปแบบการเรียนการสอนในรูปแบบที่เด็กสามารถเรียนรู้ได้ด้วยตัวเอง และสนุกไปกับมัน 
เลยทำให้เกิดโครงการ เกมเสริมทักษะวิทยาการคำนวณ ขึ้นมา

$\>$โครงการ หนีออกจากวังวน (Escape) จะเน้นด้านการคิดเชิงคำนวณ 
สามารถช่วยแก้ไขปัญหาการที่นักเรียนไม่ได้เรียนวิชาวิทยาการคำนวณอย่างที่ควรเป็น 
โดยทางผู้พัฒนาจะนำ block code เข้ามาใช้ในตัวเกมเพื่ออิงตามหลักสูตรวิชาวิทยาการคำนวณของกระทรวงศึกษาธิการ 
ที่ได้มีการเรียนการสอนโดยใช้ scratch ที่เป็น block code เหมือนกัน 
โดยตัวเกมจะเป็นการที่ผู้เล่น (นักเรียน) นำ block code มาวางวางเพื่อบังคับตัวละครในเกมเพื่อแก้ไขปัญหาในแต่ละด่านที่ได้ออกแบบอิงตามแบบฝึกหัดหนังสือของ 
สสวท. วิชาวิทยาการคำนวณ
% \enskip ภาควิชาวิศวกรรมคอมพิวเตอร์จึงได้จัดทำต้นแบบรูปเล่มรายงานโดยใช้ระบบจัดเตรียมเอกสาร
% \LaTeX{} เพื่อช่วยให้นักศึกษาเขียนรายงานได้อย่างสะดวกและรวดเร็วมากยิ่งขึ้น
\end{abstractTH}

\begin{abstract}
The abstract would be placed here. It usually does not exceed 350 words
long (not counting the heading), and must not take up more than one (1) page
(even if fewer than 350 words long).

Make sure your abstract sits inside the \texttt{abstract} environment.
\end{abstract}

\iffalse
\begin{dedication}
This document is dedicated to all Chiang Mai University students.

Dedication page is optional.
\end{dedication}
\fi % \iffalse

\begin{acknowledgments}
โครงการ หนีออกจากวังวน (Escape) 
นี้จะประสบความสำเร็จไม่ได้หากไม่ได้รับการสนับสนุนจาก 
คณะ วิศวกรรมศาสตร์ มหาวิทยาลัยเชียงใหม่ 
ในด้านการอำนวยความสะดวกและสถานที่ในการพัฒนาโปรแกรม 
อาจารย์ ชินวัตร อิศราดิสัยกุล ที่ให้คำปรึกษาในการพัฒนาโปรแกรม 
ทีมผู้ร่วมพัฒนาที่คอยช่วยเหลือกันในการพัฒนาโปรแกรม 
นากจากนี้ทางคณะผู้พัฒนายังได้รับกำลังใจจากบิดมารดาพี่น้องอาจารย์และ
เพื่อนของผู้พัฒนา อำนวยความสะดวกและให้คำปรึกษาด้านโปรแกรม 
% \texttt{acknowledgment} environment.

\acksign{2020}{5}{25}
\end{acknowledgments}%
\fi % \ifproject

\contentspage

\ifproject
\figurelistpage

\tablelistpage
\fi % \ifproject

% \abbrlist % this page is optional

% \symlist % this page is optional

% \preface % this section is optional
