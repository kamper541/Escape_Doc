\newcounter{choice}
\renewcommand\thechoice{\Alph{choice}}
\newcommand\choicelabel{\thechoice.}

\newenvironment{choices}%
  {\list{\choicelabel}%
     {\usecounter{choice}\def\makelabel##1{\hss\llap{##1}}%
       \settowidth{\leftmargin}{W.\hskip\labelsep\hskip 2.5em}%
       \def\choice{%
         \item
       } % choice
       \labelwidth\leftmargin\advance\labelwidth-\labelsep
       \topsep=0pt
       \partopsep=0pt
     }%
  }%
  {\endlist}

% A
\chapter{การหาค่าความเที่ยงตรงของแบบสอบถาม IOC: Index of item objective congruence}

การหาค่า IOC ของผู้เชี่ยวชาญ จากการให้ผู้เชี่ยวชาญตรวจสอบแบบสอบถามการวิจัย IOC คือค่าความเที่ยงตรงของแบบสอบถามหรือ ค่าสอดคล้อง
ระหว่างข้อคำถามกับวัตถุประสงค์หรือเนื้อหา ปกติแล้วจะให้ผู้เชี่ยวชาญตรวจสอบ ตั้งแต่ 3 คนขึ้นไปในการตรวจสอบ

\section{เกณฑ์การพิจารณาข้อคำถาม}
\begin{enumerate}
    \item ให้คะแนน +1 ถ้าแน่ใจว่าข้อคำถามวัดได้ตรงตามวัตถุประสงค์
    \item ให้คะแนน 0 ถ้าไม่แน่ใจว่าข้อคำถามวัดได้ตรงตามวัตถุประสงค์
    \item ให้คะแนน -1 ถ้าแน่ใจว่าข้อคำถามสัดได้ไม่ตรงตามวัตถุประสงค์
\end{enumerate}

\section{การหาค่า IOC ในแต่ละหัวข้อ}
วิธีหาค่าของ IOC คือการนำคะแนนรวมของผู้เชี่ยวชาญในแต่ละหัวข้อมาหารด้วยจำนวณของผู้เชี่ยวชาญ

\section{เกณฑ์ความเที่ยงตรง}
\begin{enumerate}
    \item ข้อคำถามที่มีค่า IOC ตั้งแต่ 0.50--1.00 มีค่าความเที่ยงตรงที่ใช้ได้
    \item ข้อคำถามที่มีค่า IOC ต่ำกว่า 0.50 ต้องปรับปรุง ยังใช้ไม่ได้
\end{enumerate}

% ฺB
\chapter{รายชื่อของผู้เข้าร่วมทำการทดสอบ}
รายละเอียดดังต่อไปนี้จะประกอบไปด้วย ชื่อ--สกุล และข้อมูลการศึกษา/ทำงานของนักเรียนชั้นประถมศึกษาปีที่ 3--6 โรงเรียนบ้านออนใต้ และโรงเรียนบ้านโฮ้ง และผู้เชี่ยวชาญที่ร่วมทำทดสอบตัวเกมเสริมทักษะวิชาวิทยาการคำนวณ
\section{รายชื่อและข้อมูลการทำงานของผู้เชี่ยวชาญ}
\begin{center}
    \begin{tabular}{ |p{5cm}|p{5cm}| }
        \hline
        ชื่อนามสกุล & บริษัทที่สังกัด\\
        \hline
        นางสาว ณัชชา สุวรรณยิก & Thinknet\\
        \hline
        นางสาว ณัฐวิภา ไชยกันย์ & Unalog\\
        \hline
        นางสาว ศศิวิมล บัวคำปัน & 29Next\\
        \hline
    \end{tabular}
\end{center}
\section{รายชื่อของนักเรียนชั้นประถมศึกษาปีที่ 3--6 ที่เข้าร่วมการทดสอบ}
\begin{center}
    \begin{tabular}{ |p{3cm}|p{4cm}|p{3cm}| }
        \hline
        เลขประจำตัวนักเรียน & ชื่อ-สกุล & ระดับชั้น\\
        \hline
        2645 & ด.ช. กล้าณรงค์ & ประถมศึกษาปีที่ 3\\
        \hline
        2646 & ด.ช. ขณดล พาทยโกศล & ประถมศึกษาปีที่ 3\\
        \hline
        2649 & ด.ช. ณัฐพล ศรคำ & ประถมศึกษาปีที่ 3\\
        \hline
        2650 & ด.ช. มณเฑียร ไชยชนะ & ประถมศึกษาปีที่ 3\\
        \hline
        2651 & ด.ช. มณตรี ไชยชนะ & ประถมศึกษาปีที่ 3\\
        \hline
        2644 & ด.ญ. กัญญาพักต์ สินโพธิ์ & ประถมศึกษาปีที่ 3\\
        \hline
        2647 & ด.ญ. ชุติภา ผ่องใส & ประถมศึกษาปีที่ 3\\
        \hline
        2657 & ด.ช. กิตติศักดิ์ ปัญญาดี & ประถมศึกษาปีที่ 3\\
        \hline
        2658 & ด.ช. ธนพล สุขตุ้ย & ประถมศึกษาปีที่ 3\\
        \hline
        2660 & ด.ช. สุขสันต์ ไม่ปรากฎ & ประถมศึกษาปีที่ 3\\
        \hline
        2665 & ด.ช. หลู่ & ประถมศึกษาปีที่ 3\\
        \hline
        2683 & ด.ญ. จ่ามแสง & ประถมศึกษาปีที่ 3\\
        \hline
        928 & ด.ช. กิตติพันธุ์ ขันธสีมา & ประถมศึกษาปีที่ 3\\
        \hline
        985 & ด.ช. นพณัฐ ใจดา & ประถมศึกษาปีที่ 3\\
        \hline
        986 & ด.ช. ปวริศร์ เปี้ยวงค์ & ประถมศึกษาปีที่ 3\\
        \hline
        988 & ด.ช. วันชนะ เมืองตา & ประถมศึกษาปีที่ 3\\
        \hline
        989 & ด.ช. อนุพันธุ์ นำโน & ประถมศึกษาปีที่ 3\\
        \hline
        990 & ด.ญ. กมลทิพย์ กันธาทิพย์ & ประถมศึกษาปีที่ 3\\
        \hline
        991 & ด.ญ. ชโณทัย เกอะภัย & ประถมศึกษาปีที่ 3\\
        \hline
        992 & ด.ญ. ภัทรจาริน คำวัน & ประถมศึกษาปีที่ 3\\
        \hline
        993 & ด.ญ. มนัสพร รัตนจันทร์ & ประถมศึกษาปีที่ 3\\
        \hline
        2627 & ด.ช. จัรภัทร สิงห์บี้ & ประถมศึกษาปีที่ 4\\
        \hline
        2629 & ด.ช. ปิยะพงษ์ จันตา & ประถมศึกษาปีที่ 4\\
        \hline
        967 & ด.ช. ดนุสรณ์ จะกู & ประถมศึกษาปีที่ 4\\
        \hline
        969 & ด.ช. ดนุสรณ์ จะกู & ประถมศึกษาปีที่ 4\\
        \hline
        970 & ด.ช. ภัทรชัย บูรณานุสรณ์ & ประถมศึกษาปีที่ 4\\
        \hline
        971 & ด.ช. ภานุพงค์ โยคำ & ประถมศึกษาปีที่ 4\\
        \hline
        972 & ด.ช. วงศกร ไชยวงค์ & ประถมศึกษาปีที่ 4\\
        \hline
        973 & ด.ช. ศุภรัตน์ กันธง & ประถมศึกษาปีที่ 4\\
        \hline
        974 & ด.ญ. กรรณิกา จอมใจป้อ & ประถมศึกษาปีที่ 4\\
        \hline
        975 & ด.ญ. กัญญาภัค ปิรินชิน & ประถมศึกษาปีที่ 4\\
        \hline
        976 & ด.ญ. กณิกา หลำทราย & ประถมศึกษาปีที่ 4\\
        \hline
        977 & ด.ญ. จุธามาศ ปิ๊กมี & ประถมศึกษาปีที่ 4\\
        \hline
        997 & ด.ญ. สิริณภัทร ปัญญาเรือง & ประถมศึกษาปีที่ 4\\
        \hline
        1009 & ด.ญ. เต็มดวง โปธิตา & ประถมศึกษาปีที่ 4\\
        \hline
    \end{tabular}
\end{center}

\begin{center}
    \begin{tabular}{ |p{3cm}|p{4cm}|p{3cm}| }
        \hline
        เลขประจำตัวนักเรียน & ชื่อ-สกุล & ระดับชั้น\\
        \hline
        2630 & ด.ช. พรรณษา กุณาวงค์ & ประถมศึกษาปีที่ 4\\
        \hline
        2631 & ด.ช. ราชันย์ ยะอื่อ & ประถมศึกษาปีที่ 4\\
        \hline
        2632 & ด.ช. วันวิสา ยิ่งทองคำ & ประถมศึกษาปีที่ 4\\
        \hline
        2634 & ด.ช. อดิศักดิ์ ใจติขะ & ประถมศึกษาปีที่ 4\\
        \hline
        2636 & ด.ช. อ่องส่วย ไม่ปรากฎ & ประถมศึกษาปีที่ 4\\
        \hline
        2640 & ด.ช. กวนอู ส่า & ประถมศึกษาปีที่ 4\\
        \hline
        2641 & ด.ช. ต้นปี ศรีนวล & ประถมศึกษาปีที่ 4\\
        \hline
        2653 & ด.ช. ดนุพล ลุงอ่อง & ประถมศึกษาปีที่ 4\\
        \hline
        2637 & ด.ช. บัวหอม คิงคำ & ประถมศึกษาปีที่ 4\\
        \hline
        2664 & ด.ช. มล ลุงเรียง & ประถมศึกษาปีที่ 4\\
        \hline
        2628 & ด.ช. ดำรงเดช วรรณผา & ประถมศึกษาปีที่ 4\\
        \hline
        2620 & ด.ช. เกียรติศักดิ์ ใจติขะ & ประถมศึกษาปีที่ 5\\
        \hline
        2623 & ด.ญ. อภิสรา คำจา & ประถมศึกษาปีที่ 5\\
        \hline
        2626 & ด.ญ. ศิริยากร ศิริศิลป์ & ประถมศึกษาปีที่ 5\\
        \hline
        2643 & ด.ญ. ปนิดา สมฮาย & ประถมศึกษาปีที่ 5\\
        \hline
        2608 & ด.ญ. แสงหอม ลุงเจริญ & ประถมศึกษาปีที่ 5\\
        \hline
        959 & ด.ญ. บุญเจริญ ศรคำ & ประถมศึกษาปีที่ 5\\
        \hline
        960 & ด.ช. วีรเดช ใจยี & ประถมศึกษาปีที่ 5\\
        \hline
        961 & ด.ช. ภูรินท์ มะโนวงค์ & ประถมศึกษาปีที่ 5\\
        \hline
        962 & ด.ญ. วริษา มะโนวงค์ & ประถมศึกษาปีที่ 5\\
        \hline
        963 & ด.ญ. วรรณภา กันตีมูล & ประถมศึกษาปีที่ 5\\
        \hline
        964 & ด.ญ. เพ็ญนภา พวงแสง & ประถมศึกษาปีที่ 5\\
        \hline
        965 & ด.ญ. จริยาวดี รุ่งวัฒน์กิจ & ประถมศึกษาปีที่ 5\\
        \hline
        994 & ด.ญ. กรรณิการ์ เสนาคำ & ประถมศึกษาปีที่ 5\\
        \hline
        995 & ด.ช. ณรงค์ฤษธิ์ คุณยศยิ่ง & ประถมศึกษาปีที่ 5\\
        \hline
        995 & ด.ช. ณรงค์ฤษธิ์ คุณยศยิ่ง & ประถมศึกษาปีที่ 5\\
        \hline
        2604 & ด.ช. จิรภัทร ตุ้ยตา & ประถมศึกษาปีที่ 6\\
        \hline
        2606 & ด.ช. ศุภกฤต เสนแสง & ประถมศึกษาปีที่ 6\\
        \hline
        2639 & ด.ช. สี ลุงสุ & ประถมศึกษาปีที่ 6\\
        \hline
        2655 & ด.ช. เพชรนคร ศรีบุตรา & ประถมศึกษาปีที่ 6\\
        \hline
        2609 & ด.ญ. ณัฐญา พรมใจ & ประถมศึกษาปีที่ 6\\
        \hline
        2611 & ด.ญ. สุมินตรา แก้วผัด & ประถมศึกษาปีที่ 6\\
        \hline
        2612 & ด.ญ. แก้ว จำซา & ประถมศึกษาปีที่ 6\\
        \hline
        2666 & ด.ช. ธนพัฒน์ เคียงพงษ์ & ประถมศึกษาปีที่ 6\\
        \hline
        951 & ด.ช. หิรัญ ฟูคำ & ประถมศึกษาปีที่ 6\\
        \hline
        957 & ด.ช. ชวลิต ลัยญา & ประถมศึกษาปีที่ 6\\
        \hline
        954 & ด.ญ. จิตรสุภา ชัยญา & ประถมศึกษาปีที่ 6\\
        \hline
        996 & ด.ช. ณัฐพงศ์ กันธาทิพย์ & ประถมศึกษาปีที่ 6\\
        \hline
        950 & ด.ช. ปีเตอร์ ชิงกิม วี & ประถมศึกษาปีที่ 6\\
        \hline
    \end{tabular}
\end{center}

% C
\chapter{Pre-test/post-test}
Pre-test และ post-test ถูกนำมาใช้เพื่อแสดงให้เห็นถึงพัฒนาการของผู้ใช้ โดยการวัดผลก่อนและหลังการทำอะไรบางอย่าง ซึ่งในที่นี้คือการเล่นเกมเสริมทักษะวิชาวิทยาการคำนวณ
โดยตัวอย่างของ pre-test และ post-test จะอยู่ในรูปแบบของ Google Forms โดยมีจำนวนทั้งหมด 10 ข้อต่อชุด ดังนี้
\section{ตัวอย่าง pre-test}
\begin{enumerate}
    \item ข้อใดคือการคิดโดยใช้หลักการคิดเชิงคำนวณ
    \begin{enumerate}
        \item ทำความเข้าใจปัญหา เลือกวิธีการแก้ปัญหาที่ดีที่สุด เรียงลำดับขั้นตอน
        \item เรียงลำดับขั้นตอน ทำความเข้าใจปัญหา เลือกวิธีการแก้ปัญหาที่ดีที่สุด
        \item ทำความเข้าใจปัญหา เรียงลำดับขั้นตอน เลือกวิธีการแก้ปัญหาที่ดีที่สุด
        \item เลือกวิธีการแก้ปัญหาที่ดีที่สุด เรียงลำดับขั้นตอน ทำความเข้าใจปัญหา
    \end{enumerate}
    \item จงเรียงลำดับการต้มมาม่าเบื้องต้น
    \begin{itemize}
        \item ต้มน้ำในหม้อ
        \item ฉีกซองมาม่า
        \item ใส่มาม่าลงในหม้อ
        \item ใส่เครื่องปรุง
        \item ตักมาม่าใส่ถ้วย
    \end{itemize}
    \item เพราะเหตุใดเราถึงควรใช้หลักการคิดเชิงคำนวณในการแก้ปัญหา
    \begin{enumerate}
        \item แก้ไขปัญหาต่างๆ ในชีวิตได้อย่างเป็นระบบและมีขั้นตอน
        \item จดจำและบันทึกข้อมูลได้เป็นจำนวนมาก
        \item ช่วยในสามารถทำงานได้รวดเร็วขึ้น
        \item ช่วยให้ทักษะการคิดเปรียบเสมือนคอมพิวเตอร์
    \end{enumerate}
    \item หลักการคิดเชิงคำนวณสามารถนำไปประยุกต์ในสถานการณ์ได้บ้าง
    \begin{enumerate}
        \item การวางแผนจัดร้านค้า
        \item การทำน้ำผลไม้ปั่น
        \item การซัก อบ รีด เสื้อผ้า
        \item ถูกทุกข้อ
    \end{enumerate}
    \item ข้อใดบอกขั้นตอนการหุงข้าวได้ถูกต้อง
    \begin{enumerate}
        \item ตวงข้าวสาร > หุงข้าว > ล้างข้าวให้สะอาด > ตวงน้ำให้เหมาะสม
        \item ตวงข้าวสาร > ตวงน้ำให้เหมาะสม > ล้างข้าวให้สะอาด > หุงข้าว
        \item ตวงข้าวสาร > ล้างข้าวให้สะอาด > ตวงน้ำให้เหมาะสม > หุงข้าว
        \item ตวงข้าวสาร > ล้างข้าวให้สะอาด > หุงข้าว > ตวงน้ำให้เหมาะสม
    \end{enumerate}
    \item จากรูปข้อใดคือเส้นทางที่ทำให้หนูไปหาชีสก้อนสีเหลืองได้
    \begin{center}
        \includegraphics[width=5cm, height=5cm]{pic-toro/exam/cat.png}
    \end{center}
    \begin{enumerate}
        \item ขึ้น ขวา ขวา ขึ้น
        \item ขึ้น ขวา ขวา ขวา
        \item ขึ้น ขึ้น ขวา ขวา
        \item ขึ้น ลง ขวา ขวา
    \end{enumerate}
    \item จากรูปจงเขียนลูกศรนำทางให้เด็กชายไปเอากุญแจเพื่อมาเปิดกล่องสมบัติ
    \begin{center}
        \includegraphics[width=5cm, height=5cm]{pic-toro/exam/treasure.png}
    \end{center}
    \item จากรูปจงเขียนลูกศรนำทางให้เด็กชายไปเอากุญแจเพื่อมาเปิดกล่องสมบัติ
    \begin{center}
        \includegraphics[width=5cm, height=5cm]{pic-toro/exam/treasure2.png}
    \end{center}
    \item นำทางคนป่ากลับถ้ำโดยใช้ลูกศร
    \begin{center}
        \includegraphics[width=5cm, height=5cm]{pic-toro/exam/cave.png}
    \end{center}
    \item นำทางหนูไปหาชีสโดยที่ไม่ต้องเจอแมว
    \begin{center}
        \includegraphics[width=5cm, height=5cm]{pic-toro/exam/cathard.png}
    \end{center}
\end{enumerate}

\section{ตัวอย่าง post-test}
\begin{enumerate}
    \item ข้อใดกล่าวถึงหลักการคิดเชิงคำนวณได้ถูกต้อง
    \begin{enumerate}
        \item เป็นการแก้ปัญหาแบบมีลำดับขั้นตอน
        \item เป็นทักษะที่นักพัฒนาซอฟต์แวร์ต้องมี
        \item เป็นการคิดเหมือนหุ่นยนต์
        \item ข้อ 1 และ ข้อ 2 ถูกต้อง
    \end{enumerate}
    \item จงเรียงลำดับการข้ามถนนบนทางม้าลายที่มีสัญญาณไฟจราจรคนข้ามถนนให้ถูกต้อง
    \begin{itemize}
        \item ไฟจราจรสีเขียว "ข้ามได้"
        \item สังเกตสัญญาณไฟจราจรคนข้ามถนน
        \item เดินข้ามถนนตรงทางม้าลาย
        \item กดปุ่มสัญญาณไฟจราจรคนข้ามถนน
        \item ไฟจราจรสีแดง "หยุดรอ"
    \end{itemize}
    \item กระบวนการแก้ปัญหาจะต้องเริ่มจากขั้นตอนใดเป็นขั้นตอนแรก
    \begin{enumerate}
        \item ดำเนินการแก้ไข
        \item วางแผนการแก้ปัญหา
        \item ตรวจสอบและปรับปรุง
        \item วิเคราะห์และกำหนดรายละเอียดของปัญหา
    \end{enumerate}
    \item ถ้านักเรียนต้องจัดกระเป๋าเพื่อไปเที่ยว ขั้นตอนใดเรียงลำดับได้เหมาะสมที่สุด
    \begin{enumerate}
        \item อาบน้ำ > แต่งตัว > สตาร์ทรถ > เติมน้ำมัน > จัดกระเป๋า
        \item สตาร์ทรถ > เติมน้ำมัน > อาบน้ำ > จัดกระเป๋า > แต่งตัว
        \item อาบน้ำ > แต่งตัว > จัดกระเป๋าอาบน้ำ > สตาร์ทรถ > เติมน้ำมัน
        \item จัดกระเป๋า > อาบน้ำ > แต่งตัว > สตาร์ทรถ > เติมน้ำมัน
    \end{enumerate}
    \item ข้อใดเรียงลำดับการใช้งานคอมพิวเตอร์ได้อย่างเหมาะสมที่สุด
    \begin{enumerate}
        \item เสียบปลั๊ก > กดปุ่มเปิดเครื่อง > กดปุ่มเปิดหน้าจอ > ใช้งาน > กด Shutdown > ปิดหน้าจอ > ถอดปลั๊ก
        \item เสียบปลั๊ก > กดปุ่มเปิดหน้าจอ > ใช้งาน > กด Shutdown > ปิดหน้าจอ > ถอดปลั๊ก
        \item เสียบปลั๊ก > กดปุ่มเปิดเครื่อง > กดปุ่มเปิดหน้าจอ > ใช้งาน > ปิดหน้าจอ > ถอดปลั๊ก
        \item เสียบปลั๊ก > กดปุ่มเปิดหน้าจอ > กดปุ่มเปิดเครื่อง > ใช้งาน > ถอดปลั๊ก > ปิดหน้าจอ
    \end{enumerate}
    \item ช่วยผึ้งหนีหมีและเก็บดอกไม้ไปยังรังผึ้ง
    \begin{center}
        \includegraphics[width=5cm, height=5cm]{pic-toro/exam/bee.png}
    \end{center}
    \item จงลากดินสอไปหาเลข 1 2 3 ตามลำดับแล้วไปยังเส้นชัย
    \begin{center}
        \includegraphics[width=5cm, height=5cm]{pic-toro/exam/pen.png}
    \end{center}
    \item จงรดน้ำต้นไม้ให้ครบทุกต้น
    \begin{center}
        \includegraphics[width=5cm, height=5cm]{pic-toro/exam/treeeasy.png}
    \end{center}
    \item หาวิธีรดน้ำต้นไม้ให้ครบทุกต้นโดยใช้จำนวนครั้งในการเดินน้อยที่สุด
    \begin{center}
        \includegraphics[width=5cm, height=5cm]{pic-toro/exam/treemed.png}
    \end{center}
    \item รดน้ำต้นไม้ให้ครบทุกต้นโดยที่บัวรดน้ำ 1 อันรดน้ำต้นไม้ได้ 3 ครั้ง จากนั้นต้องกลับมาเติมใหม่
    \begin{center}
        \includegraphics[width=5cm, height=5cm]{pic-toro/exam/treehard.png}
    \end{center} 
\end{enumerate}

\chapter{\ifcpe คู่มือการใช้งานระบบ\else Manual\fi}

เกมแอปพลิเคชันของโครงงานนี้ จะเป็นเกมแนวอาร์เคดที่สามารถเล่นได้ทุกเพศทุกวัย 
เป็นเกมที่เล่นโดยใช้ทักษะการคิดเชิงคำนวณ เพื่อเล่นให้ผ่านในแต่ละด่าน เก็บคะแนนที่จะได้เป็นดาว แล้วปลดล็อกด่านต่อไปเรื่อยๆ 
โดยภายในเกมจะมีแผนที่อยู่ด้วยกันทั้งหมด 3 แผนที่ แต่ละแผนที่จะมีสภาพแวดล้อมที่แตกต่างกัน และมีด่านอยู่ภายในทั้งหมด 10 
ด่าน แต่ละด่านก็จะมีความยากง่ายแตกต่างกัน โดยจะมีความซับซ้อนและมีเงื่อนไขใหม่ๆ
เพิ่มเข้ามา เพื่อให้ผู้เล่นได้ใช้และพัฒนาทักษะการคิดเชิงคำนวณมากขึ้น

\section{การใช้งานพื้นฐาน}
\begin{enumerate}
    \item หน้าเมนูหลัก กดปุ่ม Start เพื่อไปยังหน้าหน้าเลือกแผนที่ที่ได้ทำการปลดล็อกแล้ว หากต้องการจะกลับมายังหน้าเมนูหลัก ให้กดปุ่ม Home
    \item หน้าเลือกแผนที่ ให้ทำการเลือกแผนที่ที่ต้องการเล่น จากทั้งหมด 3 แผนที่ โดยเริ่มต้นจะปลดล็อกเพียงแผนที่แรกเท่านั้น ผู้เล่นต้องทำการเล่นเก็บดาวเพื่อมาปลดล็อกแผนที่ถัดๆ ไป ดาวที่เราได้รับจะแสดงอยู่มุมขวาบนของหน้าจอ \CI{ที่มีรูปดาวและจำนวน}{?}
    \item หน้าเลือกด่านจะประกอบไปด้วยด่านทั้งหมด 10 ด่าน ในแต่ละด่านจะมีดาวให้เก็บ 3 ดาว ระบบจะวัดผลจากคะแนนที่ได้ในด่านนั้นให้เป็นดาว โดยต้องผ่านอย่างน้อย 1 ดาวจึงจะไปยังด่านต่อไปได้
    \item เมื่อผ่านด่านแล้ว จะมีปุ่มให้เลือกกดไปยังด่านถัดไป หากไม่ผ่านด่าน ต้องทำการเล่นอีกครั้งจนกว่าจะได้อย่างน้อย 1 ดาว
\end{enumerate}

\section{วิธีการเล่น}
\begin{enumerate}
    \item เมื่อเริ่มเกม ตัวละครจะอยู่ห่างจากจุดหมายระยะหนึ่ง ให้นำพาตัวละครไปยังจุดหมายโดยใช้หลักการคิดเชิงคำนวณในวิเคราะห์เส้นทาง 
    
\end{enumerate}

\section{สัญลักษณ์ปุ่มต่างๆภายในเกม}

