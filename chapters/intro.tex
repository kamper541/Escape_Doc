\chapter{\ifcpe บทนำ\else Introduction\fi}

\section{\ifcpe ที่มาของโครงงาน\else Project rationale\fi}
$\>$จากการที่กระทรวงศึกษาธิการได้เพิ่มหลักสูตรวิชาวิทยาการคำนวณ 
เพื่อให้เด็กนักเรียนได้มีความพร้อมในยุคเทคโนโลยีดิจิตอล 
และเป็นการเสริมความรู้ในด้านทักษะการคิดเชิงคำนวณ 
พื้นฐานด้านเทคโนโลยีดิจิทัล และ พื้นฐานการรู้เท่าทันสื่อและข่าวสาร 
ทำให้ผู้พัฒนาเล็งเห็นถึงการกระจายความรู้ที่เป็นไปได้ยากในโรงเรียนรอบนอก 
ทำให้โรงเรียนรอบนอกไม่สามารถได้รับความรู้ในรายวิชาวิทยาการคำนวณได้อย่างมีประสิทธิภาพ 
ไม่ว่าจะเป็นการที่มีบุคลากรครูที่ไม่เพียงพอ หรือบุคลากรครูที่สอนรายวิชาวิทยากรคำนวณนั้นจบไม่ตรงสาย 
ทำให้เกิดความเหลื่อมล้ำทางการศึกษาส่งผลให้เด็กนักเรียนไม่ชอบ หรือไม่รู้จักวิชาวิทยาการคำนวณว่าจริง ๆ 
แล้วมันคือวิชาอะไรเท่าที่ควร ถึงแม้ว่าทางกระทรวงศึกษาธิการได้มีการผลักดันหลักสูตรรายวิชาวิทยาการคำนวณเป็นอย่างมากแล้วก็ตาม
ทำให้ผู้พัฒนาได้เล็งเห็นว่าเราควรที่จะมีสื่ออะไรบางอย่างเพื่อเชื่อมโยงเด็กกับรายวิชาวิทยาการคำนวณด้วยกัน 
สื่อในรูปแบบที่นักเรียนมีความสนใจนั้นก็คือเกม ซึ่งจะทำให้นักเรียนได้มีปฏิสัมพันธ์กับรายวิชาวิทยาการคำนวณในรูปแบบที่นักเรียนชอบ 
และจะทำให้โรงเรียนผลิตเด็กที่มีคุณภาพเพื่อไปเป็นอนาคตของประเทศชาติต่อไป
\section{\ifcpe วัตถุประสงค์ของโครงงาน\else Objectives\fi}
\begin{enumerate}
    \item เพื่อช่วยแก้ไขปัญหาความเหลื่อมล้ำทางการศึกษาโดยเป็นเครื่องมือในการสอนโดยนำเสนอในรูปแบบของเกม
    \item เพื่อให้นักเรียนสามารถนำความรู้ที่ได้ไปปรับใช้ในรายวิชา และในชีวิตจริง และสามารถผลิตเยาวชนที่มีคุณภาพให้กับประเทศได้
    \item เพื่อเป็นการผลัดดัน และแสดงให้เห็ฯถึงความสำคัญของหลักสูตรวิชาวิทยาการคำนวณ
\end{enumerate}

\section{\ifcpe ขอบเขตของโครงงาน\else Project scope\fi}
ตัวเกมทำงานได้ในระบบ Android เท่านั้นแต่ว่าอนาคตจะมีการรอบรับระบบอื่น ๆ 
เข้ามาเช่น iOS, PC, Web App เป็นตัน 
และในแต่ละด่านนั้นจะสร้างขึ้นโดยอิงจากหลักสูตรที่อยู่ในหนังสือวิทยาการคำนวณของกระทรวงทำให้มีด่านจำกัดตามหลักสูตร 
ทำให้ด่านหลังจากนั้นผู้พัฒนาจะทำให้ตัวด่านนั้นสร้างเองโดยใช้ \texttt{Neural Network} และสามารถสร้างด่านโดยการอ่านไฟล์ text
\subsection{\ifcpe ขอบเขตด้านฮาร์ดแวร์\else Hardware scope\fi}
เป็นเกมที่ต้องเล่นบน smart phone ที่ทำงานบนระบบปฏิบัติการ Android โดยในอนาคตอาจจะมีการการพัฒนาให้สามารถเล่นบน PC , VR ได้
\subsection{\ifcpe ขอบเขตด้านซอฟต์แวร์\else Software scope\fi}
ตัวเกมต้องใช้ Android version 5.0 'Lollipop' หรือสูงกว่าและ API level 21 หรือสูงกว่า
โดยในอนาคตอาจจะมีการการพัฒนาให้สามารถเล่นบน iOS ได้
\section{\ifcpe ประโยชน์ที่ได้รับ\else Expected outcomes\fi}

\section{\ifcpe เทคโนโลยีและเครื่องมือที่ใช้\else Technology and tools\fi}

\subsection{\ifcpe เทคโนโลยีด้านฮาร์ดแวร์\else Hardware technology\fi}
ไม่มี
\subsection{\ifcpe เทคโนโลยีด้านซอฟต์แวร์\else Software technology\fi}
ตัว Game Engine ใช้ Unity3d~\cite{utb}~\cite{ud} ตัว Code block ใช้ Google Blockly\cite{gb}
ส่วนด้านแสดงผลตัว Block Code ใช้ html และ javascript โดยใช้ Google Firebase\cite{fb} ในการ Host
\section{\ifcpe แผนการดำเนินงาน\else Project plan\fi}

\begin{plan}{7}{2020}{3}{2021}
    \planitem{7}{2020}{8}{2020}{ศึกษาค้นคว้า}
    \planitem{9}{2020}{12}{2020}{พัฒนา}
    \planitem{1}{2021}{2}{2021}{ทดสอบ}
    \planitem{3}{2021}{3}{2021}{พัฒนาต่อ}
\end{plan}

\section{\ifcpe บทบาทและความรับผิดชอบ\else Roles and responsibilities\fi}
ตัวเกมจะแบ่งเป็นอยู่ 2 ส่วน คือส่วนที่อยู่บนเว็ปซึ่งจัดการด้านตัว Block Code และ ฝั่ง Unity จะจัดการด้านแสดงผล โดยแบ่งไปอีก
2 ส่วนย่อย คือ UI , Mechanic โดยบทบาทจะแบ่งเป็น
\begin{enumerate}
    \item กรวิชญ์ บัวคำปัน จัดการด้านเว็ป, เกม Mechanic, และ UI
    \item กิตติพงษ์ ไมล์หรือ จัดการด้าน UI
\end{enumerate}

\section{\ifcpe%
ผลกระทบด้านสังคม สุขภาพ ความปลอดภัย กฎหมาย และวัฒนธรรม
\else%
Impacts of this project on society, health, safety, legal, and cultural issues
\fi}

% แนวทางและโยชน์ในการประยุกต์ใช้งานโครงงานกับงานในด้านอื่นๆ รวมถึงผลกระทบในด้านสังคมและสิ่งแวดล้อมจากการใช้ความรู้ทางวิศวกรรมที่ได้
เกิดเป็นสื่อการเรียบการสอนเพื่อให้เด็กนักเรียนสามารถเข้าถึงตััววิชาวิทยาการคำนวณในรูปแบบที่เด็กชื่นชอบและ จะทำให้เด็กนั้นมีความเข้าใจ
ในรายวิชา เพื่อนำไปใช้ในการแก้ปัญหาในการเรียน หรือแม้กระทั่งปัญหาในชีวิตจริง