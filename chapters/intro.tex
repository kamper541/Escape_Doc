\chapter{\ifcpe บทนำ\else Introduction\fi}

\section{\ifcpe ที่มาของโครงงาน\else Project rationale\fi}
ในปี พ.ศ.\,2560 กระทรวงศึกษาธิการได้เพิ่มหลักสูตรวิชาวิทยาการคำนวณ~\cite{cpc} มาในรายวิชาวิทยาศาสตร์
 เพื่อให้เด็กนักเรียนได้มีความพร้อมในยุคเทคโนโลยีดิจิทัล 
และเป็นการเสริมความรู้ในด้านทักษะการคิดเชิงคำนวณ 
พื้นฐานด้านเทคโนโลยีดิจิทัล และ พื้นฐานการรู้เท่าทันสื่อและข่าวสาร

ผู้พัฒนาได้มีความสนใจในการเข้าไปศึกษาเรียนรู้เกี่ยวกับการเรียนการสอนของตัวหลักสูตรและ
จากประสบการณ์ของผู้พัฒนาที่ได้คลุกคลีกับโรงเรียนรอบนอก
ทำให้เล็งเห็นถึงการกระจายความรู้ที่เป็นไปได้ยากในโรงเรียนรอบนอก ผู้พัฒนาจึงทำการสำรวจจากการลงพื้นที่โรงเรียนรอบนอก
 ได้แก่ โรงเรียนบ้านออนใต้ โรงเรียนมิตรมวลชน และโรงเรียนบ้านดอยเต่า

จากผลการสำรวจ พบปัญหาคือ โรงเรียนรอบนอกไม่สามารถได้รับความรู้ในรายวิชาวิทยาการคำนวณได้อย่างมีประสิทธิภาพ 
ไม่ว่าจะเป็นการที่มีบุคลากรครูที่ไม่เพียงพอ หรือบุคลากรครูที่สอนรายวิชาวิทยากรคำนวณนั้นจบไม่ตรงสาย 
ทำให้เกิดความเหลื่อมล้ำทางการศึกษาส่งผลให้เด็กนักเรียนไม่ชอบ หรือไม่รู้จักวิชาวิทยาการคำนวณว่าจริงๆ 
แล้ววิชานี้คือวิชาอะไร ถึงแม้ว่าทางกระทรวงศึกษาธิการได้มีการ\CI{ผลักดันหลักสูตรรายวิชาวิทยาการคำนวณเป็นอย่างมากแล้วก็ตาม}{cite}

ในการสำรวจครั้งเดียวกันนี้ สามารถสรุปเพิ่มเติมได้ว่า เด็กนักเรียนนั้นชอบเล่นเกมที่เป็นแนว puzzle แก้ปัญหาเป็นด่านๆ ผู้พัฒนาจึงมีแนวคิดในการสร้างสื่อในรูปแบบที่นักเรียนจะสนใจ นั่นก็คือเกม
ซึ่งจะทำให้นักเรียนได้มีปฏิสัมพันธ์กับรายวิชาวิทยาการคำนวณในรูปแบบที่นักเรียนชอบ

เนื่องจากการคิดเชิงคำนวณเป็นสิ่งที่เป็นพื้นฐานในการแก้ไขปัญหาต่างๆ ในชีวิตประจำวันอยู่แล้ว จึงเกิดเป็นโครงงาน Escape นี้ เพื่อนำเสนอการเรียนการสอนวิชาวิทยาการคำนวณในรูปแบบของเกม
โดยเราจะเน้นไปที่การคิดเชิงคำนวณเป็นหลัก

\section{\ifcpe วัตถุประสงค์ของโครงงาน\else Objectives\fi}
\begin{enumerate}
    \item เพื่อช่วยแก้ไขปัญหาความเหลื่อมล้ำทางการศึกษาโดยเป็นเครื่องมือในการสอนโดยนำเสนอในรูปแบบของเกม
    \item เพื่อให้นักเรียนสามารถนำความรู้ที่ได้ไปปรับใช้ในรายวิชา และในชีวิตจริง และสามารถผลิตเยาวชนที่มีคุณภาพให้กับประเทศได้
    \item เพื่อเป็นการผลักดัน และแสดงให้เห็นถึงความสำคัญของหลักสูตรวิชาวิทยาการคำนวณ
\end{enumerate}

\section{\ifcpe ขอบเขตของโครงงาน\else Project scope\fi}
% \CIreply{ไม่ชัดเจนว่าจะทำอะไร จะไม่ทำอะไร}
ตัวเกมทำงานได้ในระบบ Android เท่านั้น แต่ว่าในอนาคตจะมีการรองรับระบบอื่นๆ 
เข้ามาเช่น iOS, PC, Web App เป็นตัน 
และในแต่ละด่านนั้นจะสร้างขึ้นโดยอิงจากหลักสูตรที่อยู่ในหนังสือวิทยาการคำนวณของกระทรวงศึกษาธิการ
โดยจะสามารถสร้างด่านเพิ่มเติมได้โดยง่ายด้วยวิธีการระบุรายละเอียดของด่านผ่าน \newline text files

\subsection{\ifcpe ขอบเขตด้านฮาร์ดแวร์\else Hardware scope\fi}
เป็นเกมที่ต้องเล่นบน smartphones ที่ทำงานบนระบบปฏิบัติการ Android โดยในอนาคตอาจจะมีการการพัฒนาให้สามารถเล่นบน iOS, PC, VR ได้

\subsection{\ifcpe ขอบเขตด้านซอฟต์แวร์\else Software scope\fi}
ตัวเกมต้องใช้ Android version 5.0 'Lollipop' หรือสูงกว่าและ API level 21 หรือสูงกว่า
โดยในอนาคตอาจจะมีการการพัฒนาให้สามารถเล่นบน iOS ได้

\section{\ifcpe ประโยชน์ที่ได้รับ\else Expected outcomes\fi}
ประโยชน์ที่ได้รับที่คาดไว้ มีอยู่ 2 ด้าน 
\begin{enumerate}
    \item นักเรียน: นักเรียนได้เรียนรู้ในตัววิชาเพื่อนำไปแก้ไขปัญหาในวิชาเรียน รวมไปถึงปัญหาในชีวิตประจำวันได้
    \item โรงเรียน: เมื่อนักเรียนสามารถแก้ไขปัญหาต่างๆ ได้ ทางโรงเรียนเองจะสามารถผลิตเด็กที่มีคุณภาพสู่สังคมได้ 
\end{enumerate}
% \CIreply{เขียนแยกทีละประเด็น}

\section{\ifcpe เทคโนโลยีและเครื่องมือที่ใช้\else Technology and tools\fi}

\subsection{\ifcpe เทคโนโลยีด้านฮาร์ดแวร์\else Hardware technology\fi}
Touchscreen บนหน้าจอของ smartphones

\subsection{\ifcpe เทคโนโลยีด้านซอฟต์แวร์\else Software technology\fi}
ตัว game engine ใช้ Unity3d~\cite{utb,ud} ตัว block code ใช้ Google Blockly~\cite{gb}
ส่วนด้านแสดงผลตัว block code ใช้ JavaScript~\cite{js} เป็นหลัก โดยใช้ localhost ในการ host

\section{\ifcpe แผนการดำเนินงาน\else Project plan\fi}
\subsection{แผนการดำเนินงานในช่วงแรก}
แผนการดำเนินงานในช่วงแรกซึ่งเป็นส่วนหนึ่งของวิชา 261491 ภาคการเรียนที่ 2 ปีการศึกษา 2563 มีรายละเอียดตามตารางที่~\ref{firstworksplan}
\begin{table}
    \begin{plan}{11}{2020}{6}{2021}
        \planitem{11}{2020}{11}{2020}{ค้นหาปัญหา และสืบค้นข้อมูลเพื่อนำมาใช้เป็นหัวข้อของโครงการ}
        \planitem{11}{2020}{11}{2020}{ศึกษาข้อมูลเกี่ยวกับการใช้งานโปรแกรมที่ใช้ในการพัฒนา}
        \planitem{12}{2020}{12}{2020}{เริ่มทดลองระบบและทดลองใช้ assets ของ Unity}
        \planitem{12}{2020}{1}{2021}{ออกแบบเกมเพลย์ของเกมและแผนที่ภายในเกม}
        \planitem{12}{2020}{1}{2021}{ออกแบบ UX/UI และการใช้งานเบื้องต้น}
        \planitem{1}{2021}{1}{2021}{ศึกษาเกี่ยวกับ Blockly เพื่อใช้สำหรับสร้าง และพัฒนา block code}
        \planitem{2}{2021}{3}{2021}{ทดสอบการใช้งานและแก้ไข block code ที่สร้างขึ้น}
        \planitem{3}{2021}{4}{2021}{เริ่มออกแบบฟังก์ชันภายในเกมและพัฒนาขึ้นมาเป็น prototype}
        \planitem{4}{2021}{4}{2021}{ศึกษาเกี่ยวกับระบบ hosting ด้วย Firebase ให้สามารถใช้งานกับ Unity ได้}
        \planitem{5}{2021}{6}{2021}{พัฒนา block code และทดสอบระบบกับ Firebase และ Unity ให้เสร็จสิ้น}
        \planitem{6}{2021}{6}{2021}{ทำ UX/UI และ องค์ประกอบภายในเกม}
        \planitem{6}{2021}{6}{2021}{ทดสอบเกมโดยรวม ทำการค้นหา bugs และแก้ไขปัญหาที่เกิดขึ้น}
    \end{plan}
    \caption[ตารางการทำงานช่วงแรก]{ตารางการทำงานช่วงแรก}
    \label{firstworksplan}
\end{table}
\noindent 
\subsection{แผนการดำเนินงานในช่วงหลัง}
แผนการดำเนินงานในช่วงแรกซึ่งเป็นส่วนหนึ่งของวิชา 261492 ภาคการเรียนที่ 2 ปีการศึกษา 2564 มีรายละเอียดตามตารางที่~\ref{lastworksplan}
\begin{table}
    \begin{plan}{10}{2021}{3}{2022}
        \planitem{10}{2021}{11}{2021}{ศึกษาระบบการทำงานให้สามารถทำงานบน localhost}
        \planitem{11}{2021}{11}{2021}{ปรับปรุงและพัฒนา WebView ให้สามารถทำงานบน localhost ได้}
        \planitem{12}{2021}{12}{2021}{พัฒนาระบบ in-game tutorial}
        \planitem{12}{2021}{1}{2022}{ออกแบบและพัฒนา map renderer ให้สามารถสร้าง map ได้จากไฟล์ text}
        \planitem{1}{2022}{1}{2022}{ออกแบบและพัฒนาระบบการให้คะแนน}
        \planitem{1}{2022}{1}{2022}{สร้างแบบทดสอบ pre-test และ post-test ที่มีความแตกต่างกัน โดยอ้างอิงเนื้อหาจากหนังสือเรียน}
        \planitem{1}{2022}{2}{2022}{พัฒนาออกมาเป็น alpha release แล้วนำไปทดสอบกับกลุ่มผู้ใช้งานจริงเพื่อเก็บ feedback}
        \planitem{2}{2022}{2}{2022}{ปรับปรุงแก้ไขระบบจาก feedback ที่ได้รับมา}
        \planitem{2}{2022}{2}{2022}{นำไปทดสอบด้วยโมเดล IOC เพื่อวัดคุณภาพของเครื่องมือโดยผู้เชี่ยวชาญ}
        \planitem{2}{2022}{2}{2022}{ปรับปรุงแก้ไขแล้วเผยแพร่เป็น beta release}
        \planitem{2}{2022}{3}{2022}{จัดทำรายงาน}
    \end{plan}
    \caption[ตารางการทำงานช่วงหลัง]{ตารางการทำงานช่วงหลัง}
    \label{lastworksplan}
\end{table}

\section{\ifcpe บทบาทและความรับผิดชอบ\else Roles and responsibilities\fi}
ตัวเกมจะแบ่งเป็นอยู่ 2 ส่วน คือส่วนที่อยู่บนเว็บซึ่งจัดการด้านตัว Block Code และ ฝั่ง Unity จะจัดการด้านแสดงผล โดยแบ่งไปอีก
2 ส่วนย่อย คือ UI , Mechanic โดยบทบาทจะแบ่งเป็น
\begin{enumerate}
    \item กรวิชญ์ บัวคำปัน จัดการด้านเว็ป, เกม Mechanic, และ UI
    \item กิตติพงษ์ ไมล์หรือ จัดการด้าน UI
\end{enumerate}

\section{\ifcpe%
ผลกระทบด้านสังคม สุขภาพ ความปลอดภัย กฎหมาย และวัฒนธรรม
\else%
Impacts of this project on society, health, safety, legal, and cultural issues
\fi}

% แนวทางและโยชน์ในการประยุกต์ใช้งานโครงงานกับงานในด้านอื่นๆ รวมถึงผลกระทบในด้านสังคมและสิ่งแวดล้อมจากการใช้ความรู้ทางวิศวกรรมที่ได้
เกิดเป็นสื่อการเรียนการสอนเพื่อให้เด็กนักเรียนสามารถเข้าถึงตัววิชาวิทยาการคำนวณในรูปแบบ \newline ที่เด็กชื่นชอบและจะทำให้เด็กนั้นมีความเข้าใจในรายวิชา เพื่อนำไปใช้ในการแก้ปัญหาในการเรียน หรือแม้กระทั่งปัญหาในชีวิตจริง ทำให้เด็กออกไปใช้ชีวิตในโลกภายนอกอย่างมีความรู้ และมีภูมิคุ้มกันในการใช้งานสื่อและเทคโนโลยี เพื่อที่ประเทศจะได้พัฒนาต่อไปโดยคนรุ่นใหม่
