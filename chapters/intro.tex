\chapter{\ifcpe บทนำ\else Introduction\fi}

\section{\ifcpe ที่มาของโครงงาน\else Project rationale\fi}
ในปี พ.ศ. 2560 กระทรวงศึกษาธิการได้เพิ่มหลักสูตรวิชาวิทยาการคำนวณ~\cite{cpc}มาในรายวิชาวิทยาศาสตร์
 เพื่อให้เด็กนักเรียนได้มีความพร้อมในยุคเทคโนโลยีดิจิทัล 
และเป็นการเสริมความรู้ในด้านทักษะการคิดเชิงคำนวณ 
พื้นฐานด้านเทคโนโลยีดิจิทัล และ พื้นฐานการรู้เท่าทันสื่อและข่าวสาร\par

ผู้พัฒนาได้มีความสนใจในการเข้าไปศึกษาเรียนรู้เกี่ยวกับการเรียนการสอนของตัวหลักสูตรและ
จากประสบการณ์ของผู้พัฒนาที่ได้คลุกคลีกับโรงเรียนรอบนอก
ทำให้เล็งเห็นถึงการกระจายความรู้ที่เป็นไปได้ยากในโรงเรียนรอบนอก ผู้พัฒนาเลยทำการที่จะสำรวจและจากการลงพื้นที่โรงเรียนรอบนอก
 ได้แก่ โรงเรียนบ้านออนใต้ โรงเรียนมิตรมวลชน และโรงเรียนบ้านดอยเต่า\par

จากผลการสำรวจได้พบเจอกับปัญหาคือ โรงเรียนรอบนอกไม่สามารถได้รับความรู้ในรายวิชาวิทยาการคำนวณได้อย่างมีประสิทธิภาพ 
ไม่ว่าจะเป็นการที่มีบุคลากรครูที่ไม่เพียงพอ หรือบุคลากรครูที่สอนรายวิชาวิทยากรคำนวณนั้นจบไม่ตรงสาย 
ทำให้เกิดความเหลื่อมล้ำทางการศึกษาส่งผลให้เด็กนักเรียนไม่ชอบ หรือไม่รู้จักวิชาวิทยาการคำนวณว่าจริง ๆ 
แล้ววิชานี้คือวิชาอะไร ถึงแม้ว่าทางกระทรวงศึกษาธิการได้มีการผลักดันหลักสูตรรายวิชาวิทยาการคำนวณเป็นอย่างมากแล้วก็ตาม\par

จากอีกหนึ่งผลสำรวจได้ว่าเด็กนักเรียนนั้นชอบเล่นเกมที่เป็นแนว Puzzle แก้ปัญหาเป็นด่านๆ ผู้พัฒนาเลยมีความคิดว่าจะทำสื่อในรูปแบบที่นักเรียนมีความสนใจนั้นก็คือเกม
ซึ่งจะทำให้นักเรียนได้มีปฏิสัมพันธ์กับรายวิชาวิทยาการคำนวณในรูปแบบที่นักเรียนชอบ และการคิดเชิงคำนวณเป็นสิ่งที่เป็นพื้นฐานในการแก้ไขปัญหาต่างๆ ในชีวิตประจำวัน
เลยทำให้เกิดโครงงาน Escape นี้ขึ้นมาเพื่อนำเสนอการเรียนการสอนวิชาวิทยาการคำนวณในรูปแบบของเกม
โดยเราจะเน้นไปที่ การคิดเชิงคำนวณ

\section{\ifcpe วัตถุประสงค์ของโครงงาน\else Objectives\fi}
\begin{enumerate}
    \item เพื่อช่วยแก้ไขปัญหาความเหลื่อมล้ำทางการศึกษาโดยเป็นเครื่องมือในการสอนโดยนำเสนอในรูปแบบของเกม
    \item เพื่อให้นักเรียนสามารถนำความรู้ที่ได้ไปปรับใช้ในรายวิชา และในชีวิตจริง และสามารถผลิตเยาวชนที่มีคุณภาพให้กับประเทศได้
    \item เพื่อเป็นการผลักดัน และแสดงให้เห็นถึงความสำคัญของหลักสูตรวิชาวิทยาการคำนวณ
\end{enumerate}

\section{\ifcpe ขอบเขตของโครงงาน\else Project scope\fi}
% \CIreply{ไม่ชัดเจนว่าจะทำอะไร จะไม่ทำอะไร}
ตัวเกมทำงานได้ในระบบ Android เท่านั้นแต่ว่าอนาคตจะมีการรอบรับระบบอื่น ๆ 
เข้ามาเช่น iOS, PC, Web App เป็นตัน 
และในแต่ละด่านนั้นจะสร้างขึ้นโดยอิงจากหลักสูตรที่อยู่ในหนังสือวิทยาการคำนวณของกระทรวง
โดยจะสามารถสร้างด่านโดยการอ่านไฟล์ text

\subsection{\ifcpe ขอบเขตด้านฮาร์ดแวร์\else Hardware scope\fi}
เป็นเกมที่ต้องเล่นบน Smart Phone ที่ทำงานบนระบบปฏิบัติการ Android โดยในอนาคตอาจจะมีการการพัฒนาให้สามารถเล่นบน iOs, PC, VR ได้

\subsection{\ifcpe ขอบเขตด้านซอฟต์แวร์\else Software scope\fi}
ตัวเกมต้องใช้ Android version 5.0 'Lollipop' หรือสูงกว่าและ API level 21 หรือสูงกว่า
โดยในอนาคตอาจจะมีการการพัฒนาให้สามารถเล่นบน iOS ได้

\section{\ifcpe ประโยชน์ที่ได้รับ\else Expected outcomes\fi}
ประโยชน์ที่ได้รับที่คาดไว้มีอยู่ 2 ด้าน 
\begin{enumerate}
    \item นักเรียน นักเรียนได้เรียนรู้ในตัววิชาเพื่อนำไปแก้ไขปัญหาในวิชาเรียนรวมไปถึงปัญหาในชีวิตประจำวันได้
    \item โรงเรียน เมื่อนักเรียนสามารถแก้ไขปัญหาต่างๆ ได้ ทางโรงเรียนเองสามารถผลิตเด็กที่มีคุณภาพเพื่อออกไปสู่สังคัมได้ 
\end{enumerate}
% \CIreply{เขียนแยกทีละประเด็น}

\section{\ifcpe เทคโนโลยีและเครื่องมือที่ใช้\else Technology and tools\fi}

\subsection{\ifcpe เทคโนโลยีด้านฮาร์ดแวร์\else Hardware technology\fi}
Touch Screen บนหน้าจอของ Smart Phone

\subsection{\ifcpe เทคโนโลยีด้านซอฟต์แวร์\else Software technology\fi}
ตัว Game Engine ใช้ Unity3d~\cite{utb,ud} ตัว Code Block ใช้ Google Blockly~\cite{gb}
ส่วนด้านแสดงผลตัว Block Code ใช้ JavaScript\cite{js} เป็นหลัก โดยใช้ Localhost~ ในการ Host

\section{\ifcpe แผนการดำเนินงาน\else Project plan\fi}
\begin{plan}{11}{2020}{3}{2021}
    \planitem{11}{2020}{11}{2020}{ค้นหาปัญหา และ สืบค้นข้อมูลเพื่อนำมาใช้เป็นหัวข้อของโครงการ}
    \planitem{11}{2020}{11}{2020}{ศึกษาข้อมูลเกี่ยวกับการใช้งานโปรแกรมที่ใช้ในการพัฒนา}
    \planitem{12}{2020}{12}{2020}{เริ่มทดลองระบบและทดลองใช้ Asset ของ Unity}
    \planitem{12}{2020}{1}{2021}{ออกแบบเกมเพลย์ของเกมและแผนที่ภายในเกม}
    \planitem{12}{2020}{1}{2021}{ออกแบบ UX/UI และการใช้งานเบื้องต้น}
  %  \planitem{1}{2021}{1}{2021}{ทำการออกแบบแผนที่ภายในเกม}
    \planitem{1}{2021}{1}{2021}{ศึกษาเกี่ยวกับ Blockly เพื่อใช้สำหรับสร้าง และพัฒนา Block Code}
    \planitem{2}{2021}{3}{2021}{ทดสอบการใช้งานและแก้ไข Block Code ที่สร้างขึ้น}
    \planitem{3}{2021}{3}{2021}{เริ่มออกแบบฟังก์ชันภายในเกมและพัฒนาขึ้นมาเป็น Prototype}
    %\planitem{4}{2021}{5}{2021}{ศึกษาเกี่ยวกับระบบ Hosting ด้วย Localhost ให้สามารถใช้งานกับ Unity ได้}
   % \planitem{5}{2021}{5}{2021}{ทดสอบการใช้งานระบบ Hosting ของ Localhost ให้ใช้งานกับ Unity ได้}
    %\planitem{6}{2021}{9}{2021}{พัฒนา Block Code และ ทดสอบระบบกับ Localhost และ Unity ให้เสร็จสิ้น}
    %\planitem{6}{2021}{9}{2021}{ทำ UX/UI และ องค์ประกอบภายในเกม}
    %\planitem{7}{2021}{10}{2021}{ทดสอบเกมโดยรวม ทำการค้นหาบัคและแก้ไขปัญหาที่เกิดขึ้น}


\end{plan}

\begin{plan}{11}{2021}{3}{2022}
    \planitem{11}{2021}{11}{2021}{ปรับปรุงและพัฒนา WebView ให้สามารถทำงานบน Localhost ได้}
    \planitem{12}{2021}{12}{2021}{พัฒนาระบบ In-game Tutorial}
    \planitem{12}{2021}{1}{2022}{ออกแบบและพัฒนา Map Renderer ให้สามารถสร้าง Map ได้จากไฟล์ text}
    \planitem{1}{2022}{1}{2022}{ออกแบบและพัฒนาระบบการให้คะแนน}
    \planitem{1}{2022}{1}{2022}{สร้างแบบทดสอบ Pre-test Post-test ที่มีความแตกต่างกัน โดยอ้างอิงเนื้อหาจากหนังสือเรียน}
    \planitem{1}{2022}{2}{2022}{พัฒนาออกมาเป็น Alpha Release แล้วนำไปทดสอบกับกลุ่มผู้ใช้งานจริงเพื่อเก็บ Feedback}
    \planitem{2}{2022}{2}{2022}{ปรับปรุงแก้ไขระบบจาก Feedback ที่ได้รับมา}
    \planitem{2}{2022}{2}{2022}{นำไปทดสอบด้วยโมดเดล IOC เพื่อวัดคุณภาพของเครื่องมือโดยผู้เชี่ยวชาญ}
    \planitem{2}{2022}{2}{2022}{ปรับปรุงแก้ไขแล้ว Release Beta}
    \planitem{2}{2022}{3}{2022}{จัดทำรายงาน}



\end{plan}

\section{\ifcpe บทบาทและความรับผิดชอบ\else Roles and responsibilities\fi}
ตัวเกมจะแบ่งเป็นอยู่ 2 ส่วน คือส่วนที่อยู่บนเว็บซึ่งจัดการด้านตัว Block Code และ ฝั่ง Unity จะจัดการด้านแสดงผล โดยแบ่งไปอีก
2 ส่วนย่อย คือ UI , Mechanic โดยบทบาทจะแบ่งเป็น
\begin{enumerate}
    \item กรวิชญ์ บัวคำปัน จัดการด้านเว็ป, เกม Mechanic, และ UI
    \item กิตติพงษ์ ไมล์หรือ จัดการด้าน UI
\end{enumerate}

\section{\ifcpe%
ผลกระทบด้านสังคม สุขภาพ ความปลอดภัย กฎหมาย และวัฒนธรรม
\else%
Impacts of this project on society, health, safety, legal, and cultural issues
\fi}

% แนวทางและโยชน์ในการประยุกต์ใช้งานโครงงานกับงานในด้านอื่นๆ รวมถึงผลกระทบในด้านสังคมและสิ่งแวดล้อมจากการใช้ความรู้ทางวิศวกรรมที่ได้
เกิดเป็นสื่อการเรียบการสอนเพื่อให้เด็กนักเรียนสามารถเข้าถึงตััววิชาวิทยาการคำนวณในรูปแบบที่เด็กชื่นชอบและ
จะทำให้เด็กนั้นมีความเข้าใจในรายวิชาเพื่อนำไปใช้ในการแก้ปัญหาในการเรียน หรือแม้กระทั่งปัญหาในชีวิตจริง ทำให้เด็กออกไปใช้ชีวิตในโลกภายนอกอย่างมีความรู้และ
ภูมิคุ้มกันในสื่อและเทคโนโลยี เพื่อที่ประเทศจะได้พัฒนาต่อไปโดยคนรุ่นใหม่